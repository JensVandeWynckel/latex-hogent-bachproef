%%=============================================================================
%% Appendix
%%=============================================================================

\chapter{Appendix}
\label{app:appendix}

De configuratie in tabel~\ref{tab:custom-eu-card} wordt gebruikt om in Netskope creditcardnummers te detecteren aan de hand van vooraf gedefinieerde identificatiegegevens.  
Het script in Lijst~\ref{lst:dlp-script} automatiseert het verzenden van testgegevens via \texttt{https://dlptoolbox\-.com} en Netskope om de detectie van creditcardnummers te testen. 
Dit script wordt niet gebruikt in het onderzoek doordat \texttt{dlptoolbox} regelmatig bij veel verkeer de gebruiker afschermde.
Het script~\ref{lst:send-request} automatiseert het verzenden van testgegevens naar een zelf opgestelde Ngrok webapplicatie, die de testgegevens verwerkt.
Tabel~\ref{tab:eval-criteria-performance} geeft een overzicht van hoe de performantie van de Netskope agent mogelijks geëvalueerd kan worden. 
Deze tabel wordt deels gebruikt, maar was moeilijk vergelijkbaar door de verschillende tools die gebruikt wordt.
\textcite{Netskope2025Utilization} stelt vast dat de agent een zeer lage impact heeft op de performantie van de computer.

% \section{Custom DLP-regel: Detectie van creditcardnummers}
% \label{sec:custom-dlp-regel}

\begin{table}[h]
    \centering
    \small
    \scriptsize
    \begin{tabular}{p{4cm} p{10cm}}
        \toprule
        \textbf{Regel} & EU \-- PCI \-- Card Number \\
        \midrule
        \textbf{Type} & DLP Rule \\
        \textbf{Expressie} & \texttt{P0 OR P1 OR P2 OR P3 OR P4 OR P5 OR P6 OR P7 OR P8 OR P9 OR P10 OR P11 OR P12 OR (P13 OR P14 OR P15 OR P16) OR (P17 OR P18 OR P19 OR P20) OR (P21 OR P22 OR P23 OR P24) OR (P25 OR P26 OR P27 OR P28 OR P29 OR P30 OR P31) OR (P32 OR P33 OR P34 OR P35) OR (P36 OR P37 OR P38)} \\
        \textbf{Predefined Identifiers} & 
        (P0) \-- Card Numbers (all) \newline
        (P1) \-- Card Numbers (Major Networks; with spaces) \newline
        (P2) \-- Card Numbers (Major Networks; all) \newline
        (P3) \-- Card Numbers (Major Networks; unformatted) \newline
        (P4) \-- Card Numbers (Major Networks; with dots) \newline
        (P5) \-- Card Numbers (Major Networks; with hyphens) \newline
        (P6) \-- Card Numbers (Maestro) \newline
        (P7) \-- Deposit Account Numbers (BE) \newline
        (P8) \-- Card Numbers (Mastercard) \newline
        (P9) \-- Card Numbers (VISA) \newline
        (P10) \-- Defunct Card Numbers (Bankcard) \newline
        (P11) \-- Defunct Card Numbers (all) \newline
        (P12) \-- Domestic Card Numbers (all) \newline
        (P13) \-- Card Numbers (19 digits, all) \newline
        (P14) \-- Card Numbers (19 digits, unformatted) \newline
        (P15) \-- Card Numbers (19 digits, with hyphens) \newline
        (P16) \-- Card Numbers (19 digits, with spaces) \newline
        (P17) \-- Card Numbers (18 digits, all) \newline
        (P18) \-- Card Numbers (18 digits, unformatted) \newline
        (P19) \-- Card Numbers (18 digits, with hyphens) \newline
        (P20) \-- Card Numbers (18 digits, with spaces) \newline
        (P21) \-- Card Numbers (17 digits, all) \newline
        (P22) \-- Card Numbers (17 digits, unformatted) \newline
        (P23) \-- Card Numbers (17 digits, with hyphens) \newline
        (P24) \-- Card Numbers (17 digits, with spaces) \newline
        (P25) \-- Card Numbers (16 digits, all) \newline
        (P26) \-- Card Numbers (16 digits, unformatted) \newline
        (P27) \-- Card Numbers (16 digits, with dots) \newline
        (P28) \-- Card Numbers (16 digits, with hyphens) \newline
        (P29) \-- Card Numbers (16 digits, with non-breaking spaces) \newline
        (P30) \-- Card Numbers (16 digits, with soft hyphens) \newline
        (P31) \-- Card Numbers (16 digits, with spaces) \newline
        (P32) \-- Card Numbers (15 digits, all) \newline
        (P33) \-- Card Numbers (15 digits, unformatted) \newline
        (P34) \-- Card Numbers (15 digits, with hyphens) \newline
        (P35) \-- Card Numbers (15 digits, with spaces) \\
        \textbf{Gebruikte techniek} & Vooraf gedefinieerde DLP identificatiegegevens gebruikt rond creditcardnummers \\
        \textbf{Bron(nen)} & Null \\
        \textbf{Threshold} & Low: 1 \quad Medium: 2 \quad High: 5 \quad Critical: 9 \\
        \bottomrule
    \end{tabular}
    \caption{Custom DLP-regel: Detectie van creditcardnummers}
    \label{tab:custom-eu-card}
\end{table}

% \section{Custom DLP-regel: Detectie van creditcardnummers in combinatie met banktermen}

\begin{table}[h]
    \centering
    \small
    \scriptsize
    \begin{tabular}{p{4cm} p{10cm}}
        \toprule
        \textbf{Regel} & BE \-- Card Number with Terms \\
        \midrule
        \textbf{Type} & DLP Rule \\
        \textbf{Expressie} & \texttt{(P0 NEAR P36) OR (P0 NEAR P37) OR (P0 NEAR P38) OR (P0 NEAR P39) OR (P0 NEAR P40) OR (P1 NEAR P36) OR (P1 NEAR P37) OR (P1 NEAR P38) OR (P1 NEAR P39) OR (P1 NEAR P40) OR (P2 NEAR P36) OR (P2 NEAR P37) OR (P2 NEAR P38) OR (P2 NEAR P39) OR (P2 NEAR P40) OR (P3 NEAR P36) OR (P3 NEAR P37) OR (P3 NEAR P38) OR (P3 NEAR P39) OR (P3 NEAR P40) OR (P4 NEAR P36) OR (P4 NEAR P37) OR (P4 NEAR P38) OR (P4 NEAR P39) OR (P4 NEAR P40) OR (P5 NEAR P36) OR (P5 NEAR P37) OR (P5 NEAR P38) OR (P5 NEAR P39) OR (P5 NEAR P40) OR (P6 NEAR P36) OR (P6 NEAR P37) OR (P6 NEAR P38) OR (P6 NEAR P39) OR (P6 NEAR P40) OR \newline
        ... \newline
        (P35 NEAR P36) OR (P35 NEAR P37) OR (P35 NEAR P38) OR (P35 NEAR P39) OR (P35 NEAR P40)} \\
        \textbf{Predefined Identifiers} & 
        \textbf{Hetzelfde als bij vorige gedefinieerde tabel}\ref{tab:custom-eu-card} \newline
        (P36) \-- Card Number Terms (English) \newline
        (P37) \-- Card Number Terms (Spanish) \newline
        (P38) \-- Card Security Terms (English) \newline
        (P39) \-- Bank Account Number Terms (English) \newline
        (P40) \-- Bank Account Terms (BE) \\
        \textbf{Gebruikte techniek} & Combinatie van vooraf gedefinieerde DLP identificatiegegevens rond creditcardnummers met woordenlijsten van Belgische termen. \\
        \textbf{Bron(nen)} & Null \\
        \textbf{Threshold} & Low: 1 \quad Medium: 2 \quad High: 5 \quad Critical: 9 \\
        \bottomrule
    \end{tabular}
    \caption{Custom DLP-regel: Detectie van creditcardnummers in combinatie met banktermen}
    \label{tab:custom-eu-card-terms}
\end{table}


% \section{Selenium-script voor batchgewijze verzending van testdata}
% \label{sec:selenium-script}

\begin{lstlisting}[style=custompython,caption={Selenium-script voor batchgewijze verzending van testdata},label={lst:dlp-script}, captionpos=b]
    from selenium import webdriver
    from selenium.webdriver.common.by import By
    from selenium.webdriver.chrome.options import Options
    from selenium.webdriver.support.ui import WebDriverWait
    from selenium.webdriver.support import expected_conditions as EC
    import time
    
    # config
    URL = "https://dlptoolbox.com/postmaster.aspx"
    INPUT_FILE = "dlp_testdata.csv"
    ROWS_PER_BATCH = 10
    WAIT = 5
    
    # setup Chrome
    options = Options()
    options.add_argument("--headless")
    driver = webdriver.Chrome(options=options)
    wait = WebDriverWait(driver, 10)
    
    driver.get(URL)
    iframe = wait.until(EC.presence_of_element_located((By.ID, "ifrPOST")))
    driver.switch_to.frame(iframe)
    
    with open(INPUT_FILE, 'r') as f:
        lines = [line.strip() for line in f if line.strip()]
    
    for i in range(0, len(lines), ROWS_PER_BATCH):
        batch = "\n".join(lines[i:i+ROWS_PER_BATCH])
        print(f"Versturen batch {i//ROWS_PER_BATCH + 1}:\n{batch}")
    
        textarea = wait.until(EC.presence_of_element_located((By.ID, "clientDataClear")))
        textarea.clear()
        textarea.send_keys(batch)
    
        launch_btn = wait.until(EC.element_to_be_clickable((By.ID, "btnLaunch")))
        launch_btn.click()
        time.sleep(WAIT)
    
        leaked_box = driver.find_element(By.ID, "clientDataLeaked")
        leaked_data = leaked_box.get_attribute("value")
        print(f"Gelekte data ({len(leaked_data)} tekens):\n{leaked_data}")
        print("—" * 50)
    
        try:
            reset_btn = wait.until(EC.element_to_be_clickable((By.ID, "btnReset")))
            reset_btn.click()
            time.sleep(1)
        except Exception as e:
            print(f"[!] Reset mislukt: {e}")
    
    driver.quit()
\end{lstlisting}

\begin{lstlisting}[style=custompython,caption={Selenium WebDriver script voor automatisering van gegevensverwerking variabelen},label={lst:send-request}, captionpos=b]
    import csv
    import sys
    import time
    import os
    import signal
    import pyperclip
    from datetime import datetime
    from selenium import webdriver
    from selenium.webdriver.common.by import By
    from selenium.webdriver.chrome.options import Options
    from selenium.webdriver.support.ui import WebDriverWait
    from selenium.webdriver.support import expected_conditions as EC
    from selenium.webdriver.common.action_chains import ActionChains
    from selenium.webdriver.common.keys import Keys

    csv.field_size_limit(sys.maxsize)
    URL = "https://<hash>.ngrok-free.app/"
    SUMMARY_FILE = "emails_summary.csv"
    EMAILS_FILE = "emails.csv"
    LAST_SENT_FILE = "last_sent.txt"
    LOG_FILE = "log.csv"
    WAIT = 1
    RETRY_LIMIT = 1
\end{lstlsting}

\begin{lstlisting}[style=custompython,caption={Selenium WebDriver script voor automatisering van gegevensverwerking},label={lst:send-request-2}, captionpos=b]
interrupted = False
def handle_sigint(signum, frame):
    global interrupted
    interrupted = True
    print("\nOnderbroken door gebruiker (CTRL+C)...")
signal.signal(signal.SIGINT, handle_sigint)

# CHROME SETUP (stealth)
options = Options()
# options.add_argument("--headless")
options.add_argument("--disable-gpu")
options.add_argument("--disable-blink-features=AutomationControlled")
options.add_experimental_option("excludeSwitches", ["enable-automation"])
options.add_experimental_option("useAutomationExtension", False)

driver = webdriver.Chrome(options=options)
driver.execute_cdp_cmd('Page.addScriptToEvaluateOnNewDocument', {
    'source': "Object.defineProperty(navigator, 'webdriver', { get: () => undefined });"
})
wait = WebDriverWait(driver, 6)

def bypass_ngrok_warning():
    try:
        btn = wait.until(EC.element_to_be_clickable(
            (By.XPATH, "//button[contains(text(),'Visit Site')]")
        ))
        print("Ngrok warning…")
        btn.click()
        time.sleep(2)
    except:
        pass

def log_email_send(file_key, content):
    ts = datetime.now().strftime("%-m/%-d/%y %-I:%M %p")
    size = len(content.encode('utf-8'))
    header = not os.path.exists(LOG_FILE)
    with open(LOG_FILE, "a", newline="", encoding="utf-8") as f:
        w = csv.writer(f)
        if header:
            w.writerow(["timestamp","file","size_bytes"])
        w.writerow([ts, file_key, size])

# laad e-mails in
emails_dict = {}
with open(EMAILS_FILE, "r", encoding="utf-8") as f:
    for row in csv.DictReader(f):
        emails_dict[row["file"].strip()] = row["message"].strip()

to_send = []
with open(SUMMARY_FILE, "r", encoding="utf-8") as f:
    for row in csv.DictReader(f):
        fk = row["file"].strip()
        if fk in emails_dict:
            to_send.append(fk)

# laatste verzonden key
start = 0
if os.path.exists(LAST_SENT_FILE):
    with open(LAST_SENT_FILE, "r") as f:
        last = f.read().strip()
        if last in to_send:
            start = to_send.index(last) + 1

total_sent = 0
failed = []

# functie om te verzenden
def try_send(content):
    for attempt in range(RETRY_LIMIT + 1):
        try:
            # 1 Vul clipboard via pyperclip
            pyperclip.copy(content)

            # 2 Focus op textarea & paste via Cmd+V
            ta = wait.until(EC.presence_of_element_located((By.ID, "sensitive_data")))
            driver.execute_script("arguments[0].scrollIntoView(true);", ta)
            ta.click()
            time.sleep(0.1)
            ActionChains(driver) \
                .key_down(Keys.COMMAND) \
                .send_keys('v') \
                .key_up(Keys.COMMAND) \
                .perform()

            # 3 Klik Launch
            btn = wait.until(EC.element_to_be_clickable((By.XPATH, "//button[text()='Launch']")))
            driver.execute_script("arguments[0].scrollIntoView(true);", btn)
            ActionChains(driver).move_to_element(btn).click().perform()

            # 4 Wacht op resultaat
            wait.until(EC.presence_of_element_located((By.ID, "result")))
            return True

        except Exception:
            if attempt < RETRY_LIMIT:
                print("Fout, herlaad en probeer opnieuw…")
                driver.get(URL)
                bypass_ngrok_warning()
                time.sleep(1)
            else:
                return False

# verzend e-mails
driver.get(URL)
bypass_ngrok_warning()
print(f"Start vanaf index {start} ({to_send[start] if to_send else '–'})")

try:
    for idx, fk in enumerate(to_send[start:], start=start + 1):
        if interrupted:
            break

        print(f"\n[{idx}/{len(to_send)}] {fk}")
        ok = try_send(emails_dict[fk])
        if ok:
            print("Succes")
            total_sent += 1
            with open(LAST_SENT_FILE, "w") as f:
                f.write(fk)
            log_email_send(fk, emails_dict[fk])
            time.sleep(WAIT)
        else:
            print("Mislukt:", fk)
            failed.append(fk)
finally:
    driver.quit()
    print("\nSAMENVATTING")
    print("Verstuurd:", total_sent)
    print("Mislukt:", len(failed))
    if failed:
        print("Mislukte keys:", ", ".join(failed))
    if total_sent:
        print("Laatste succesvol:", to_send[start + total_sent - 1])
    sys.exit(0)
\end{lstlisting}

\begin{lstlisting}[style=custompython,caption={Python-script voor het categoriseren en samenvatten van e-mails in een CSV-bestand},label={lst:e-mail-summary}, captionpos=b]
import os
import polars as pl
import tiktoken
from tqdm import tqdm
import csv

csv_file_path = "datasets/emails.csv"
output_csv = "emails_summary.csv"
encoding_name = "cl100k_base"

# categorisatie
encoding = tiktoken.get_encoding(encoding_name)

def count_tokens(text: str) -> int:
    try:
        return len(encoding.encode(text))
    except Exception:
        return 0

# lees het CSV-bestand in batches
batch_size = 50_000
reader = pl.read_csv(csv_file_path, has_header=True, encoding="utf8-lossy", truncate_ragged_lines=True)

# check kolomnamen
assert set(reader.columns) == {"file", "message"}, "Unexpected CSV structure. Verwachte kolommen: file, message"

# data prepareren
data = reader.select(["file", "message"])

# output-bestand
with open(output_csv, "w", newline="", encoding="utf-8") as out_file:
    writer = csv.writer(out_file)
    writer.writerow(["file", "tokens", "char_len"])

    for row in tqdm(data.iter_rows(), total=data.height, desc="Verwerken"):
        file, msg = row
        if msg is None:
            continue
        token_count = count_tokens(msg)
        char_len = len(msg)
        writer.writerow([file, token_count, char_len])
\end{lstlisting}

\begin{table}[h]
    \centering
    \small
    \scriptsize
    \begin{tabular}{
        >{\raggedright\arraybackslash}p{3.5cm} 
        >{\raggedright\arraybackslash}p{4cm} 
        >{\raggedright\arraybackslash}p{2.5cm} 
        >{\raggedright\arraybackslash}p{4cm}
    }
        \toprule
        \textbf{Evaluatiecriterium} & \textbf{Tool} & \textbf{Type} & \textbf{Toelichting / Alternatief} \\ 
        \midrule
        \gls{cpu}-gebruik & Activity Monitor & GUI & Basis realtime overzicht, geen logging \\
                    & htop & CLI & Real-time overzicht per proces, logging mogelijk \\
                    & dstat & CLI & Combineert \gls{cpu}, RAM, netwerk en disk in één tool \\
        \midrule
        RAM-gebruik & Activity Monitor & GUI & Beperkt overzicht per proces \\
                    & htop & CLI & Gedetailleerd geheugenverbruik per proces \\
        \midrule
        Netwerkimpact & iPerf3 & CLI & Meet upload/download throughput, vereist server \\
                      & ping / traceroute & CLI & Latency en route-analyse \\
                      & Wireshark & GUI & Inspectie op pakketniveau, diepgaand netwerkverkeer \\
        \midrule
        Disk I/O & iostat & CLI & Meet disk throughput, latency, queue depth \\
                 & fs\_usage & CLI & File-level tracing van systeemprocessen \\
        \bottomrule
    \end{tabular}
    \caption{Overzicht evaluatiecriteria en bijhorende tools voor performantieanalyse (macOS)}
    \label{tab:eval-criteria-performance}
\end{table}