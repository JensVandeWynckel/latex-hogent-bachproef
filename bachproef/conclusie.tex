%%=============================================================================
%% Conclusie
%%=============================================================================

\chapter{\IfLanguageName{dutch}{Conclusie}{Conclusion}}%
\label{ch:conclusie}

% TODO: Trek een duidelijke conclusie, in de vorm van een antwoord op de
% onderzoeksvra(a)g(en). Wat was jouw bijdrage aan het onderzoeksdomein en
% hoe biedt dit meerwaarde aan het vakgebied/doelgroep? 
% Reflecteer kritisch over het resultaat. In Engelse teksten wordt deze sectie
% ``Discussion'' genoemd. Had je deze uitkomst verwacht? Zijn hier zaken die nog
% niet duidelijk zijn?
% Heeft het onderzoek geleid tot nieuwe vragen die uitnodigen tot verder 
%onderzoek?

% \lipsum[76-80]


Deze bachelorproef heeft de mogelijkheden van Netskope Data Loss Prevention onderzocht ter bescherming van confidentiële data binnen de Belgische regelgeving.
Door gebruik te maken van vooraf- en eigen gedefinieerde DLP-regels, samen met identificatoren voor dataclassificatie, 
slaagt Netskope in het detecteren van confidentiële data zoals \gls{pii}- en \gls{pci}-gegevens in verschillende omgevingen, waaronder \textbf{\gls{saas}}, \textbf{web}, \textbf{\gls{iaas}}, \textbf{e-mail} en \textbf{endpoints}.
% De centrale onderzoeksvraag werd beantwoord door een proof of concept op te zetten in een gecontroleerde testomgeving bij Evolane,
De centrale onderzoeksvraag werd beantwoord door het opzetten van een proof of concept in een gecontroleerde testomgeving bij Evolane,
waarbij vooral gekeken werd naar de \textbf{correctheid} van de agent en de \textbf{effectiviteit} van de detectie.
De resultaten van \textit{Enron's} dataset tonen aan dat Netskope \textbf{correct reageerde} op risico’s, waarbij een groot aantal incidenten gerelateerd waren aan realistische datapatronen zoals naam- en e-mailadrescombinaties.
Hoewel de evaluatie vooral de vooraf gedefinieerde DLP-regelset \texttt{DLP-Predefined} registreerde, werden de thresholds geoptimaliseerd naar de context van \textbf{Evolane}.
Het gehele onderzoek werd uitgevoerd met aandacht voor de Belgische en Europese wetgeving en bijkomende richtlijnen, zoals de \textbf{\gls{avg}}, \textbf{\gls{nis2}} en \textbf{\gls{pcidss}}, \textbf{\gls{dora}}, \textbf{\gls{iso}} sen \textbf{Schrems II}.
De onderzoeksdoelstellingen zijn behaald en de proof of concept heeft een meerwaarde geleverd als basis voor DLP-oplossingen binnen Belgische organisaties.



% Op basis van het uitgevoerde onderzoek blijkt dat Netskope een veelzijdige en effectieve oplossing biedt voor het beschermen van vertrouwelijke gegevens binnen Belgische organisaties. 
% Door een combinatie van vooraf gedefinieerde en op maat gemaakte DLP-regels, aangevuld met gerichte dataclassificatie, 
% slaagt Netskope erin om gevoelige informatie zoals PII- en PCI-gegevens accuraat te detecteren in uiteenlopende omgevingen — waaronder SaaS, Web, IaaS, e-mail en endpoints. 
% De centrale onderzoeksvraag werd beantwoord door de implementatie van een proof of concept in een gecontroleerde testomgeving bij Evolane. 
% Daarbij werd zowel de performantie van de agent als de effectiviteit van de detectie grondig geëvalueerd.

% De resultaten tonen aan dat Netskope in staat is om correct te reageren op risico’s, waarbij een overweldigend aantal incidenten gerelateerd was aan realistische datapatronen, 
% zoals naam- en e-mailadrescombinaties. Ook werd een juridisch kader geïntegreerd in de configuratie van de DLP-regels, met aandacht voor AVG-, NIS2- en PCI DSS-compliance. 
% Het toevoegen van Belgische identificatiekenmerken zoals het rijksregisternummer verhoogde bovendien de relevantie van de oplossing voor de lokale context.

% Hoewel de validatie met een beperkte interne testgroep enkele beperkingen kende, werd dit ruimschoots gecompenseerd door de analyse van de grootschalige Enron-maildataset. 
% De combinatie van functionele evaluatie en juridische afstemming toont aan dat Netskope niet enkel technisch performant is, maar ook inzetbaar is als compliant DLP-systeem voor Belgische organisaties.

% Reflecterend kan worden gesteld dat de onderzoeksdoelstellingen werden behaald en dat het opgezette onderzoek een duidelijke meerwaarde levert aan zowel Evolane als andere organisaties die overwegen een DLP-oplossing te implementeren. 
% De PoC leidde bovendien tot waardevolle inzichten in het optimaliseren van regelsets en drempelwaarden, afgestemd op contextspecifieke risico’s. Wat onverwacht bleek, 
% was de lage classificatiegraad van ‘kritieke’ incidenten, wat wijst op de noodzaak om standaardconfiguraties kritisch te evalueren.

% \section{\IfLanguageName{dutch}{Onderzoeksvraag}{Research question}}
% \label{sec:onderzoeksvraag}

% Deze bachelorproef onderzocht de mogelijkheden van Netskope Data Loss Prevention voor het beschermen van confidentiële gegevens binnen de Belgische regelgeving,
% met als doel DLP-oplossingen te \textbf{ontwerpen}, \textbf{implementeren} en \textbf{evalueren} voor Evolane.
% Om de centrale onderzoeksvraag~\ref{sec:onderzoeksvraag} te beantwoorden, zijn vijf deelvragen geformuleerd en beantwoord.

% \textbf{Centrale vraag:} \textit{“Hoe kan een op Netskope gebaseerde DLP-oplossing worden ontworpen en geïmplementeerd om vertrouwelijke gegevens te beschermen en te voldoen aan de Belgische regelgeving?”}

% \begin{enumerate}
%   \item[\textbf{1.}] \emph{Welke mogelijkheden biedt Netskope (\gls{sse}) platform voor DLP in de context van vertrouwelijke gegevensbescherming?}
%     Netskope \gls{sse} platform biedt veel mogelijkheden voor DLP aan, waaronder \texttt{Real-Time Protection}-regels~\ref{fig:netskope_rules} en dataclassificerende DLP-profielen~\ref{tab:custom-dlp-profielen}.
%     De \gls{sse} architectuur ondersteunt \textit{data-in-motion}, \textit{data-at-rest} en \textit{data-in-use} op verschillende omgevingen zoals \gls{saas}~\ref{subsubsec:saas-poc}, Web~\ref{subsubsec:web-poc}, \gls{iaas}~\ref{subsubsec:iaas-poc}, E-mail~\ref{subsubsec:e-mail-poc} en Endpoint~\ref{subsubsec:endpoint-poc}.  

%   \item[\textbf{2.}] \emph{Hoe kunnen regelsets en dataclassificatie in Netskope DLP worden afgestemd op de Belgische wetgeving, zoals de \gls{avg} en \gls{nis2}-richt\-lijn?}
%     Netskope DLP biedt vooraf gedefinieerde profielen en identificatiegegevens die voldoen aan de de \gls{avg} wetgeving en de NIS2-richtlijn.
%     Verder zijn ook eigen gedefinieerde regelsets aangemaakt voor Belgische PII-gegevens zoals het rijksregisternummer samen met de meeste voorkomende Belgische achternamen.
%     De afstemming van deze regels gebeurde door thresholds~\ref{tab:risico_thresholds} in te stellen, die zijn geoptimaliseerd aan de hand van \textbf{waarschijnlijkheid} en \textbf{impact} van de confidentiële gegevens voor Evolane~\ref{sec:toelichting_scores}.
%     Aangezien tijdens de evaluatie van de PoC de standaard thresholds van Netskope prioriteit hadden, was slechts 0,01 \% van de incidenten geclassificeerd als \emph{Critical}.

%   \item[\textbf{3.}] \emph{Welke technieken en methoden kunnen worden toegepast om \gls{pii}- en \gls{pci}-gegevens effectief te detecteren en te beschermen binnen het Netskope-platform?}
%     Tijdens de proof of concept zijn DLP-regels aangemaakt vanuit eigen- en vooraf gedefinieerde identificatiegegevens. 
%     Deze regels detecteren PII-gegevens zoals namen, e-mailadressen, rijksregisternummers~\ref{tab:custom-be-id} en Belgische betalingsgegevens~\ref{tab:custom-eu-card},~\ref{tab:custom-eu-card-terms}.
%     Tijdens de evaluatie van de PoC werden e-mails van het Amerikaanse bedrijf \textit{Enron} geanalyseerd, waarbij de \texttt{EU-Name-e-mail (narrow)} en \texttt{Name-E-mail} regels samen goed waren voor \textbf{253.507} overtredingen (> 82 \% van het totaal).
%     Deze evaluatie stond in om de DLP-regels te valideren op een realistische dataset, om aan te tonen dat Netskope DLP effectief is in het detecteren van confidentiële gegevens.

%   \item[\textbf{4.}] \emph{Hoe kan een PoC voor Netskope DLP worden opgezet in een testomgeving om de effectiviteit van de oplossing te evalueren?} 
%     De PoC voor DLP werd opgezet op de omgeving van Evolane, waarbij de Netskope agent werd geïnstalleerd op de \textit{macOS}-werkstations van 4 medewerkers.
%     Aangezien dit aantal beperkt was, werd de \textit{Enron}-maildataset gebruikt om de DLP-regels te valideren en te evalueren. 
%     Voor de evaluatie werd een confusion matrix opgesteld op basis van \textbf{100} steekproefmails, waarbij 50 e-mails volgens Netskope geen confidentiële gegevens bevatten en 50 e-mails die wel confidentiële gegevens bevatten.
%     Verder werd de perfomantie van de Netskope agent gemonitord via de \textit{macOS Activity Monitor}, waarbij geen merkbare vertraging werd vastgesteld.  

%   \item[\textbf{5.}] \emph{Welke juridische en technische normen moeten worden meegenomen bij het ontwerpen van een DLP-oplossing voor een Belgische organisatie, en hoe kan Netskope aan deze eisen voldoen?}  
%     Netskope DLP voldoet aan de Belgische en Europese juridische normen zoals de \gls{avg}, \gls{nis2} en \gls{pcidss}.
%     De implementatie van Netskope DLP houdt rekening met deze regelgeving door al de toegevoegde eigen- en vooraf gedefinieerde DLP-regels hierop te baseren.
%     Aan de hand van \textbf{geëncrypteerde logs}, die Netskope bijhoudt, wordt eerer ook voldaan aan al de toegepaste wetgevingen en richtlijnen.
%     De incidenten worden geclassificeerd op basis van de \textbf{thresholds}, het oorspronkelijke bericht van het incident kan niet worden bekeken door de Netskope beheerders.
% \end{enumerate}

