%%=============================================================================
%% Conclusie
%%=============================================================================

\chapter{\IfLanguageName{dutch}{Discussie}{Discussion}}%
\label{ch:discussie}

% \section{\IfLanguageName{dutch}{Toekomstig onderzoek}{Future research}}
% \label{sec:toekomstig-onderzoek}
% dit heb ik niet onderzocht, ik heb me enkel hierop gericht. stand van zaken, waar zit er een gat?

\paragraph{Alternatieven DLP-regels}

https://www.b2bpay.co/list-banks-europe

\paragraph{Alternatieven datasets}

\textit{CICIDS 2017} dataset.

\paragraph{Verder onderzoek}

Andere alternatieve tools, die gericht zijn op \textit{MacOS}-systemen, staan beschreven in Tabel~\ref{tab:eval-criteria-performance}.
Zie \ref{sec:eva-performantie} voor meer informatie over de performantie van de Netskope agent.

\paragraph{Onderbouwing van de risico-analysemethodologie}

\textcite{Zadeh2023} % EVOLANE JENS https://chatgpt.com/c/6829e2e1-262c-8010-8b7a-a362d5f19a38

% De combinatie van waarschijnlijkheid en impact in een kwantitatieve risicomatrix is gebaseerd op bestaande wetenschappelijke methodes die zich bewezen hebben in cybersecurity-onderzoek. 
% Zo beschrijven \textcite{Zadeh2023} een gestructureerd raamwerk waarin datalekken uit reële incidenten binnen S\&P 500-organisaties 
% geanalyseerd worden aan de hand van zowel de ernst van de impact als de kans van voorkomen. 
% In hun studie gebruiken zij een \textit{likelihood-impact matrix} als besluitvormingshulpmiddel, waarbij verschillende breach types, zoals hacking en verlies/diefstal van draagbare media, 
% worden gekwalificeerd op basis van hun geobserveerde impact en waarschijnlijkheid. 

% Deze benadering sluit nauw aan bij de methodologie die in deze bachelorproef wordt toegepast. 
% Ook hier wordt een matrix gehanteerd met scores van 1 (laag) tot 5 (hoog) voor zowel impact als waarschijnlijkheid. 
% In tegenstelling tot Zadeh et al., die de matrix op basis van incidenttypes invullen, is in deze studie gekozen om de analyse toe te passen op combinaties van datatypes (zoals naam + wachtwoord of naam + bankrekeningnummer). 
% Deze focus maakt de methode beter inzetbaar bij het opstellen van een DLP-configuratie, waar inhoudelijke dataprofielen centraal staan.

% De studie van Zadeh et al.\ ondersteunt dus de validiteit van een impact-waarschijnlijkheid-gebaseerde benadering en onderbouwt het gebruik van een gestructureerde risico-inschaling per informatiecategorie.
