%%=============================================================================
%% Evaluatie
%%=============================================================================

\chapter{\IfLanguageName{dutch}{Evaluatie}{Evaluation}}%
\label{ch:evaluatie}


\section{\IfLanguageName{dutch}{Evaluatie Criteria}{Evaluation Criteria}}
\label{sec:evaluatie-criteria}

Zoals besproken in Hoofdstuk \ref{ch:stand-van-zaken} zijn er verschillende evaluatiecriteria die we kunnen gebruiken om de resultaten van de proof of concept te evalueren. 
Deze criteria zijn onderverdeeld in vier hoofdcategorieën: \textit{functionaliteit}, \textit{correctheid}, \textit{performantie} en \textit{gebruiksvriendelijkheid}. 
De evaluatieperiode loopt van 22 april 2025 tot 22 mei 2025. 
Werknemers van Evolane zullen hun feedback meegeven tijdens deze periode, zodat false positives geleidelijk aan verminderen. 
Voor geen foutieve evaluatie mee te geven, zal de confusion matrix \ref{tab:confusion_matrix} bestaan uit resultaten van de laatste week. 
In deze week zullen eerder gedetecteerde false positives opnieuw getest worden om te zien of het resultaat verschild. 


\subsection{\IfLanguageName{dutch}{Functionaliteit}{Functionality}}
\label{sec:functionaliteit}

De functionaliteit van de proof of concept heeft volgende evaluatiecriteria:

\begin{table}[h]
    \centering
    \small
    \begin{tabular}{p{4cm} p{6cm} p{6cm}}
        \toprule
        \textbf{Evaluatiecriteria} & \textbf{Omschrijving} & \textbf{Bron} \\
        \midrule
        

    \end{tabular}
    \caption{\IfLanguageName{dutch}{Evaluatiecriteria voor functionaliteit}{Evaluation criteria for functionality}}
    \label{tab:eval-criteria-functionality}
\end{table}

\subsection{\IfLanguageName{dutch}{Correctheid}{Correctness}}
\label{sec:correctheid}

Confusion matrix

\subsection{\IfLanguageName{dutch}{Performantie}{Performance}}
\label{sec:performantie}

% https://chatgpt.com/c/67f97c2a-0020-8013-b19f-6df460292b38

% - CPU-gebruik
% - RAM-gebruik
% - Netwerkimpact (upload/download/latency)
% - Mogelijk disk I/O (indien relevant)

% - zonder Netskope actief
% - met Netskope actief maar idle
% - met Netskope actief én DLP-policy actief (bijv. bij upload van gevoelige data)

De metingen van de performantie gebeuren met behulp van de tools beschreven in \ref{sec:evaluatie-criteria-performance}. 
Deze tools zijn vooral gericht op \textit{MacOS}-systemen, aangezien de Netskope agent vooral op deze besturingssystemen draait binnen de organisatie. 

Voor de performantie-evaluatie van de Netskope agent werd gebruik gemaakt van zowel grafische als command-line gebaseerde tools die compatibel zijn met macOS. Tools werden geselecteerd op basis van nauwkeurigheid, loggingmogelijkheden en hun vermogen om relevante systeem- en netwerkinformatie te capteren.

De tabel hieronder geeft een overzicht van de gebruikte tools per evaluatiecriterium, met vermelding van type en eventuele alternatieven.

\textbf{HIER MAG WEL WAT VAN WEG. ZAL NIETS VAN RESULTAAT OPLEVEREN}

\begin{table}[h]
    \centering
    \small
    \begin{tabular}{
        >{\raggedright\arraybackslash}p{3.5cm} 
        >{\raggedright\arraybackslash}p{4cm} 
        >{\raggedright\arraybackslash}p{2.5cm} 
        >{\raggedright\arraybackslash}p{4cm}
    }
        \toprule
        \textbf{Evaluatiecriterium} & \textbf{Tool} & \textbf{Type} & \textbf{Toelichting / Alternatief} \\ 
        \midrule
        CPU-gebruik & Activity Monitor & GUI & Basis realtime overzicht, geen logging \\
                    & htop & CLI & Real-time overzicht per proces, logging mogelijk \\
                    & dstat & CLI & Combineert CPU, RAM, netwerk en disk in één tool \\
        \midrule
        RAM-gebruik & Activity Monitor & GUI & Beperkt overzicht per proces \\
                    & htop & CLI & Gedetailleerd geheugenverbruik per proces \\
        \midrule
        Netwerkimpact & iPerf3 & CLI & Meet upload/download throughput, vereist server \\
                      & ping / traceroute & CLI & Latency en route-analyse \\
                      & Wireshark & GUI & Inspectie op pakketniveau, diepgaand netwerkverkeer \\
        \midrule
        Disk I/O & iostat & CLI & Meet disk throughput, latency, queue depth \\
                 & fs\_usage & CLI & File-level tracing van systeemprocessen \\
        \bottomrule
    \end{tabular}
    \caption{Overzicht evaluatiecriteria en bijhorende tools voor performantieanalyse (macOS)}
    \label{tab:eval-criteria-performance}
\end{table}


% \begin{table}[h]
%     \centering
%     \small
%     \begin{tabular}{p{4cm} p{6cm} p{6cm}}
%         \toprule
%         \textbf{Evaluatiecriteria} & \textbf{Tool(s)} & \textbf{Alternatief} \\ 
%         \midrule
%         CPU-gebruik & \textit{Activity Monitor} & \textit{htop, sysctl, iostat, dstat} \\
%         RAM-gebruik & \textit{Activity Monitor} & \textit{htop} \\
%         Netwerkimpact (Throughput en latency) & \textit{Activity Monitor} & \textit{iPerf3, Wireshark} \\
%         Disk I/O & \textit{Activity Monitor} & \textit{iostat} \\
%         \bottomrule
%     \end{tabular}
%     \caption{\IfLanguageName{dutch}{Evaluatiecriteria en tools voor performantie}{Evaluation criteria and tools for performance}}
%     \label{tab:eval-criteria-performance}
% \end{table}

\subsubsection{\IfLanguageName{dutch}{Testscenario's}{Test scenarios}}
\label{sec:testscenario's}

Tabel \ref{tab:test-scenarios-performance} geeft een overzicht van de testscenario's die uitgevoerd zijn om de performantie van de Netskope agent te evalueren. 
De testscenario's zijn ontworpen om de impact van de Netskope agent op de performantie van de computer te meten in verschillende situaties. 

\begin{table}[h]
    \centering
    \small
    \begin{tabular}{p{2cm} p{6cm} p{8cm}}
        \toprule
        \textbf{Scenario} & \textbf{Netskope} & \textbf{Beschrijving} \\
        \midrule
        \textbf{1} & niet actief & Computer draait op normale manier zonder Netskope \\
        \textbf{2} & actief maar idle & DLP-Policy is niet actief \\
        \textbf{3} & actief, samen met DLP-Policy & DLP-Policy is actief maar er wordt geen data verwerkt \\
        \textbf{4} & actief, en DLP-Policy wordt gestimuleerd & Confidentiële data wordt verwerkt terwijl de DLP-Policy actief is \\
        \bottomrule
    \end{tabular}
    \caption{\IfLanguageName{dutch}{Testscenario's voor performantie}{Test scenarios for performance}}
    \label{tab:test-scenarios-performance}
\end{table}


\subsection{\IfLanguageName{dutch}{Gebruiksvriendelijkheid}{Usability}}
\label{sec:gebruiksvriendelijkheid}

De gebruiksvriendelijkheid van de proof of concept bevat volgende evaluatiecriteria: 
\textit{Configuratiegemak}, \textit{Documentatie}, \textit{Gebruiksgemak} en \textit{Ondersteuning}. 
De beoordeling van deze criteria gebeurt aan de hand van feedback van eindgebruikers bij Evolane en mijn persoonlijke ervaring tijdens het configureren van de DLP-regels. 
De volgende centrale vragen vormen de basis voor de evaluatie van de gebruiksvriendelijkheid:

\begin{itemize}
    \item \textbf{Configuratiegemak}: Hoe eenvoudig is het om de DLP-regels te configureren?
    \item \textbf{Documentatie}: Is er voldoende documentatie beschikbaar voor de DLP-regels?
    \item \textbf{Gebruiksgemak}: Hoe eenvoudig is het om de DLP-regels te gebruiken?
    \item \textbf{Ondersteuning}: Is er voldoende ondersteuning beschikbaar voor de DLP-regels?
    \label{sec:gebruiksvriendelijkheid-criteria}
\end{itemize}

