\newglossaryentry{dlp}{
    name={DLP},
    description={Data Loss Prevention, technologie om gevoelige gegevens te beschermen tegen datalekken}
}

\newglossaryentry{pii}{
    name={PII},
    description={Personally Identifiable Information, gegevens die een persoon kunnen identificeren}
}

\newglossaryentry{pci}{
    name={PCI},
    description={Payment Card Industry, verwijst vaak naar gegevens in verband met betaalkaarten. Het wordt in dit onderzoek gebruikt om te verwijzen naar alle soorten betaalgegevens}
}

\newglossaryentry{avg}{
    name={AVG},
    description={Algemene Verordening Gegevensbescherming, Europese privacywetgeving voor de bescherming van natuurlijke personen in verband met de verwerking van persoonsgegevens en betreffende het vrije verkeer van die gegevens \autocite{eu_avg2016, eu_avg2024}}
    % description={Algemene Verordening Gegevensbescherming, Europese privacywetgeving voor de bescherming van persoonsgegevens \autocite{eu_avg2016, eu_avg2024}}
}

\newglossaryentry{pcidss}{
    name={PCI DSS},
    description={Payment Card Industry Data Security Standard, een standaard voor bescherming van kaartgegevens}
}

\newglossaryentry{nis2}{
    name={NIS2},
    description={Network and Information Security Directive versie 2, Europese richtlijn voor cybersecurity \autocite{nis2directive}}
}

% \newglossaryentry{poc}{
%     name={PoC},
%     description={Proof of Concept, een eerste prototype of conceptuele implementatie}
% }

\newglossaryentry{sse}{
    name={SSE},
    description={Secure Service Edge, een cloud-based architectuur voor beveiliging en netwerktoegang}
}

\newglossaryentry{saas}{
    name={SaaS},
    description={Software as a Service, software die via de cloud wordt aangeboden}
}

\newglossaryentry{iaas}{
    name={IaaS},
    description={Infrastructure as a Service, cloud-gebaseerde infrastructuurdiensten}
}

\newglossaryentry{sla}{
    name={SLA},
    description={Service Level Agreement, een overeenkomst die de verwachte serviceprestaties en verantwoordelijkheden tussen een dienstverlener en een klant vastlegt}
}

\newglossaryentry{tls}{
    name={TLS},
    description={Transport Layer Security, een protocol voor het beveiligen van communicatie over netwerken, de opvolger van \gls{ssl}}
}

\newglossaryentry{paas}{
    name={PaaS},
    description={Platform as a Service, cloud-gebaseerde platformdiensten voor ontwikkeling en implementatie}
}

\newglossaryentry{ccb}{
    name={CCB},
    description={Centrum voor Cybersecurity België, nationaal cyberbeveiligingscentrum}
}

\newglossaryentry{iso}{
    name={ISO/IEC 27001},
    description={\textcite{ISO2022}, internationale standaard voor informatiebeveiliging}
}

\newglossaryentry{isms}{
    name={ISMS},
    description={Information Security Management System, een systeem voor het beheren van informatiebeveiliging}
}

\newglossaryentry{it}{
    name={IT},
    description={Information Technology, informatie- en communicatietechnologie}
}

\newglossaryentry{dpa}{
    name={DPA},
    description={Data Processing Agreement, een verwerkersovereenkomst tussen partijen die persoonsgegevens uitwisselen}
}

\newglossaryentry{kpi}{
    name={KPI},
    description={Key Performance Indicator, een meetbare waarde die aangeeft hoe goed een organisatie presteert ten opzichte van haar gestelde doelen}
}

\newglossaryentry{cpu}{
    name={CPU},
    description={Central Processing Unit, de hoofdprocessor van een computersysteem}
}

\newglossaryentry{ml}{
    name={ML},
    description={Machine Learning, een tak van kunstmatige intelligentie die systemen in staat stelt om te leren van gegevens}
}

\newglossaryentry{lm}{
    name={LM},
    description={Language Model, een model dat de waarschijnlijkheid van een reeks woorden voorspelt}
}

\newglossaryentry{nlp}{
    name={NLP},
    description={Natural Language Processing, een tak van kunstmatige intelligentie die zich richt op de interactie tussen computers en menselijke taal}
}

\newglossaryentry{csv}{
    name={CSV},
    description={Comma-Separated Values, een bestandsformaat voor het opslaan van gegevens in een tabelvorm}
}

\newglossaryentry{enisa}{
    name={ENISA},
    description={European Union Agency for Cybersecurity, dit agentschap ondersteunt de EU-lidstaten in cyberbeveiliging}
}

\newglossaryentry{cvv}{
    name={CVV},
    description={Card Verification Value, een beveiligingscode op een betaalkaart}
}


\newglossaryentry{pan}{
    name={PAN},
    description={Primary Account Number, het primaire rekeningnummer op een betaalkaart}
}

\newglossaryentry{iban}{
    name={IBAN},
    description={International Bank Account Number, een internationaal gestandaardiseerd rekeningnummer}
}

\newglossaryentry{casb}{
    name={CASB},
    description={Cloud Access Security Broker, een beveiligingsoplossing voor cloudtoepassingen}
}

\newglossaryentry{ztna}{
    name={ZTNA},
    description={Zero Trust Network Access, een beveiligingsmodel dat geen standaardvertrouwen biedt aan gebruikers of apparaten}
}

\newglossaryentry{siem}{
    name={SIEM},
    description={Security Information and Event Management, een systeem voor het verzamelen en analyseren van beveiligingsgegevens}
}

\newglossaryentry{swg}{
    name={SWG},
    description={Secure Web Gateway, een beveiligingsoplossing die webverkeer beschermt tegen bedreigingen}
}

\newglossaryentry{sase}{
    name={SASE},
    description={Secure Access Service Edge, een netwerkarchitectuur die beveiliging en netwerkfunctionaliteit combineert in de cloud}
}

\newglossaryentry{sad}{
    name={SAD},
    description={Sensitive Data, gevoelige gegevens die bescherming nodig hebben, zoals persoonsgegevens, financiële gegevens of klanteninformatie}
}

\newglossaryentry{chd}{
    name={CHD},
    description={Cardholder Data, gegevens van de kaarthouder die bescherming nodig hebben, zoals het kaartnummer, de naam van de kaarthouder en de vervaldatum van de kaart}
}


\newglossaryentry{dcs}{
    name={DCS},
    description={Data Classification System, een systeem voor het classificeren van gegevens op basis van gevoeligheid}
}

\newglossaryentry{npa}{
    name={NPA},
    description={Netskope Private Access, een oplossing voor veilige toegang tot interne applicaties zonder een \gls{vpn}}
}

\newglossaryentry{vpn}{
    name={VPN},
    description={Virtual Private Network, een technologie die een veilige verbinding over het internet mogelijk maakt}
}

\newglossaryentry{san}{
    name={SAN},
    description={Storage Area Network, een netwerk dat opslagbronnen verbindt}
}

\newglossaryentry{usb}{
    name={USB},
    description={Universal Serial Bus, een standaard voor aansluiting van randapparatuur op computers}
}

\newglossaryentry{isp}{
    name={ISP},
    description={Internet Service Provider, een bedrijf dat internettoegang biedt}
}

\newglossaryentry{dora}{
    name={DORA},
    description={Digital Operational Resilience Act, een EU-wetgeving voor operationele veerkracht in de financiële sector}
}

\newglossaryentry{url}{
    name={URL},
    description={Uniform Resource Locator, een adres voor een bron op het internet, ook wel bekend als een webadres of hostname}
}

\newglossaryentry{ssl}{
    name={SSL},
    description={Secure Sockets Layer, een protocol voor het beveiligen van communicatie over het internet, nu vervangen door \gls{tls}}
}

\newglossaryentry{https}{
    name={HTTPS},
    description={Hypertext Transfer Protocol Secure, een beveiligde versie van HTTP die \gls{ssl}/\gls{tls} gebruikt voor encryptie}
}

\newglossaryentry{jwt}{
    name={JWT},
    description={JSON Web Token, een open standaard voor het veilig uitwisselen van informatie tussen partijen als een JSON-object}
}

\newglossaryentry{nda}{
    name={NDA},
    description={Non-Disclosure Agreement, een geheimhoudingsovereenkomst tussen partijen om vertrouwelijke informatie te beschermen}
}

\newglossaryentry{ssh}{
    name={SSH},
    description={Secure Shell, een protocol voor veilige communicatie tussen computers over een netwerk}
}

\newglossaryentry{api}{
    name={API},
    description={Application Programming Interface, een set van regels en protocollen voor het bouwen en integreren van softwaretoepassingen}
}


\newglossaryentry{cve}{
    name={CVE},
    description={Common Vulnerabilities and Exposures, een lijst van bekende kwetsbaarheden in software en systemen}
}

\newglossaryentry{owasp}{
    name={OWASP},
    description={Open Web Application Security Project, een organisatie die zich richt op het verbeteren van de beveiliging van software}
}

\newglossaryentry{apikey}{
    name={API\--sleutels},
    description={Een unieke sleutel die wordt gebruikt om toegang te krijgen tot een Application Programming Interface (\gls{api}) en om de identiteit van de gebruiker te verifiëren}
}

\newglossaryentry{iso5}{
    name={ISO/IEC 27005},
    description={\textcite{IOSIEC2022}, een internationale standaard voor risicobeheer in informatiebeveiliging}
}

\newglossaryentry{nist}{
    name={NIST SP 800\--30 Rev. 1},
    description={\textcite{NIST800-30} Special Publication 800\--30 Revision 1, een richtlijn voor risicobeheer in informatiebeveiliging}
}

\newglossaryentry{naw}{
    name={NAW},
    description={Naam, Adres, Woonplaats, vaak gebruikt in context van persoonsgegevens}
}

\newglossaryentry{cvc}{
    name={CVC},
    description={Card Verification Code, een beveiligingscode op een betaalkaart, vergelijkbaar met \gls{cvv}}
}

\newglossaryentry{niss}{
    name={NISS},
    description={uniek identificatienummer van een natuurlijk persoon bij de Belgische sociale zekerheid. Voor inwoners van België is het identiek aan het rijksregisternummer}
}

\newglossaryentry{sso}{
    name={SSO},
    description={Single Sign-On, een authenticatiemethode die gebruikers in staat stelt om met één set inloggegevens toegang te krijgen tot meerdere applicaties of systemen}
}

\newglossaryentry{mfa}{
    name={MFA},
    description={Multi-Factor Authentication, een beveiligingsmethode die meerdere verificatiefactoren vereist voor toegang tot een systeem of applicatie}
}

\newglossaryentry{sspm}{
    name={SSPM},
    description={\gls{saas} Security Posture Management, een proces voor het beheren van de beveiligingspositie van \gls{saas}-toepassingen}
}

\newglossaryentry{uci}{
    name={UCI},
    description={User Confidence Index (gebruikersvertrouwensindex), met deze index wordt het vertrouwen van gebruikers in de beveiliging van een systeem of applicatie gemeten} 
}

% \title{How to create a glossary}
% \author{}
% \date{}