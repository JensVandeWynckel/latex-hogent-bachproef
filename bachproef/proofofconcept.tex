%%=============================================================================
%% Proof of Concept
%%=============================================================================

\chapter{\IfLanguageName{dutch}{Proof of Concept}{Proof of Concept}}%
\label{ch:proofofconcept}

\section{\IfLanguageName{dutch}{DLP-Regels}{DLP Rules}}
\label{sec:DLP rules}

Deze proof of concept (PoC) richt zich op het implementeren van Data Loss Prevention (DLP) regels binnen de Netskope omgeving.
Voor dit onderzoek wordt Netskope \textbf{126.0.0.0} gebruikt, de meest recente versie van Netskope op het moment van schrijven.

\subsection{\IfLanguageName{dutch}{Vooraf gedefinieerde regelsets}{Predefined Rulesets}}
\label{subsec:vooraf-gedefinieerde-regelsets}

In tabel~\ref{tab:Netskope regelsets} staan de vooraf gedefinieerde regelsets van Netskope die zullen worden gebruikt in dit onderzoek. 
Deze regelsets moeten voldoen aan de Belgische en Europese wetgeving, maar aangezien ze niet aanpasbaar zijn, zullen deze regelsets niet verder worden aangepast. 
Mocht de vooraf gedefinieerde regelset niet voldoen aan de opgestelde evaluatiecriteria~\ref{sec:correctheid}, dan zal een nieuwe regelset worden gedefinieerd met andere thresholds. 
Deze nieuwe regelset zal een aanvullende rol krijgen om samen met de vooraf gedefinieerde regelsets te worden gebruikt.

\begin{table}[H]
    \centering
    \small
    \scriptsize
    \begin{tabular}{p{3cm}p{5cm}p{7cm}}
        \toprule
        \textbf{Wetgeving} & \textbf{Netskope Regelsets} & \textbf{Toelichting} \\
        \midrule
        \gls{avg} (GDPR) & 
        EU General Data Protection Regulation (GDPR) \newline 
        GDPR (narrow) & 
        Detectie van persoonsgegevens volgens de \gls{avg}. Narrow zal strikter en specifieker de gegevens detecteren. \\
        
        \gls{pcidss} en \gls{dora} & 
        Payment Card Industry Data Security Standard. PCI-DSS \newline 
        DLP-PCI \newline 
        Best Practice PCI DLP Profile & 
        Detectie van creditcardgegevens, betalingsinformatie en andere financiële data. \\
        
        \gls{iso} & 
        Best Practice Confidential Tag Profile \newline 
        Best Practices ML-Based Detections & 
        Geen specifieke regelset, maar focus op governance, risicobeheer en vertrouwelijkheid. \\

        \gls{nis2} & 
        Custom policies + logging & 
        Richt op incidentdetectie en rapportering voor kritieke infrastructuur. \\
        \bottomrule
    \end{tabular}
    \caption{Mapping van wetgeving naar Netskope DLP-regelsets}
    \label{tab:Netskope regelsets}
\end{table}


\begin{table}[H]
    \centering
    \small
    \scriptsize
    \begin{tabular}{p{3cm}p{5cm}p{6cm}}
        \toprule
        \textbf{DLP-profiel} & \textbf{Gebruikte regels (predefined \& custom)} & \textbf{Toelichting / Doel} \\
        \midrule
        Persoonsgegevens Basis &
        EU-Identity-Name \newline
        EU-Name-DOB \newline
        EU-Name-Address \newline
        \textit{BE-National-ID (RRN)} &
        Basispakket voor identificatie van natuurlijke personen, inclusief Belgische rijksregisternummer-detectie. Gericht op bredere \gls{avg}-compliance. \\
        
        Gevoelige gegevens &
        EU-Identity-Health \newline
        EU-Name-Religion \newline
        EU-Name-Race \newline
        EU-Name-Political \newline
        \textit{BE-HealthData-Tag} \newline
        \textit{BE-SocialeZekerheid} &
        Detectie van gevoelige data zoals gezondheid, religie, afkomst. Gericht op risico’s met betrekking tot verwerking van bijzondere persoonsgegevens. \\
        
        Financiële gegevens &
        EU-Name-PAN \newline
        EU-Name-PAN-Exp \newline
        EU-Name-PAN-CVV \newline
        PCI-DSS \newline
        \textit{BE-IBAN, BE-BIC} &
        Beveiliging van betaal- en bankgegevens, met focus op \gls{dora}/\gls{pcidss}-compliance en Belgische bankstructuur. \\
        
        Overheidsdiensten &
        EU-Identity-Name \newline
        EU-Name-Region-Date \newline
        \textit{BE-Natinoal-ID (RRN)} &
        % \textit{BE-NIS2-incidenten} \newline
        % \textit{BE-FOD-Notificatieplicht} &
        Gericht op compliance voor openbare instellingen: incidentmelding, Belgische verplichtingen en gegevenslokalisatie. \\
        \bottomrule
    \end{tabular}
    \caption{Samengestelde DLP-profielen met gestructureerd weergegeven regels}
    \label{tab:custom-dlp-profielen}
\end{table}


% Driver License: https://learn.microsoft.com/en-us/purview/sit-defn-belgium-drivers-license-number 


% Daarnaast stelt Netskope bedrijven in staat om maatwerk DLP-profielen te definiëren voor specifieke use-cases, 
% wat flexibiliteit biedt bij de bescherming van gevoelige gegevens. Netskope integreert ook compliance-informatie in zijn Cloud Confidence Index (CCI), 
% waarin het GDPR-nalevingsniveau van cloudapplicaties wordt aangegeven. 
% Hoewel Netskope geen actief filter biedt om automatisch maatregelen te nemen op basis van het GDPR-complianceniveau van een applicatie, 
% kunnen organisaties wel beveiligingsbeleid afstemmen op basis van de Cloud Index Score.

% Een beperking binnen de huidige regelsystemen is dat Netskope geen log bijhoudt van de fysieke locatie van een datacenter waarin een cloudapplicatie wordt gehost. 
% Hierdoor is het niet mogelijk om op basis hiervan specifieke beleidsregels af te dwingen. 
% Ook ontbreekt de functionaliteit om een blokkering of waarschuwing te genereren wanneer een gebruiker binnen een bepaalde periode een grote hoeveelheid data uploadt naar 
% een niet-goedgekeurde applicatie.


% Netskope biedt een uitgebreide set vooraf gedefinieerde regelsets binnen zijn Data Loss Prevention (DLP)-functionaliteit, 
% waarmee gevoelige informatie zoals Persoonlijk Identificeerbare Informatie (PII) effectief kan worden gedetecteerd en beschermd. 
% Volgens \textcite{brouwer2021cloud} heeft Netskope een uitgebreide lijst van DLP-profielen ontwikkeld die helpen bij het identificeren van verschillende soorten PII, 
% variërend van EU-identificatiegegevens tot Singaporese identificatiegegevens en medische rapporten. 
% De DLP-profielen van Netskope zijn goed uitgebalanceerd en sluiten aan bij de regelgeving van de meeste westerse landen.

% Naast deze vooraf gedefinieerde regels biedt Netskope de mogelijkheid om maatwerk DLP-profielen te definiëren voor specifieke use-cases. 
% Verder integreert Netskope compliance-informatie in zijn Cloud Confidence Index (CCI), waarin het GDPR-nalevingsniveau van cloudapplicaties wordt aangegeven. 
% Echter, zoals \textcite{brouwer2021cloud} opmerkt, biedt Netskope geen filter om automatisch maatregelen te nemen op basis van het GDPR-complianceniveau van een applicatie.

% Daarnaast houdt Netskope geen log bij van de fysieke locatie van een datacenter waarin een cloudapplicatie wordt gehost,
%  waardoor beleid op basis van datacenterlocatie niet kan worden afgedwongen. 
%  Ook merkt \textcite{brouwer2021cloud} op dat Netskope niet de mogelijkheid biedt om een waarschuwing of blokkade te genereren wanneer een gebruiker binnen een 
%  bepaalde periode een grote hoeveelheid data uploadt naar een niet-goedgekeurde applicatie.


% Netskope allows for identification of PII with the use of DLP.
% They have defined a comprehensive list of profiles to assist in
% detecting various kinds of PII e.g. EU Identification to Singapore
% Identification as well as medical reports. Netskope has well balanced
% DLP profiles serving most western countries. Additionally, it is
% also possible to define custom DLP profiles for specific use cases.
% Netskope defines the GDPR level a cloud application offers in its
% Cloud Confidence Index but does not offer a filter to define an
% action based on users visiting an app that does not meet a set
% GDPR compliance level. Netskope does not keep a record of the
% data center an application uses and can therefore not meet this use
% case requirement. The ability to define a filter based on the overall
% Cloud Index score is possible. Lastly, it is not possible to generate
% an alert or block action if a user uploads an X amount of data within
% a certain period of time to an unsanctioned application

\subsection{\IfLanguageName{dutch}{Aangepaste regelsets}{Custom Rulesets}}
\label{subsec:poc-aangepaste-regelsets}

% <https://docs.netskope.com/en/category-definitions/> \textcite{NetskopeCatDef2025}

Dit onderzoek bevat vooral aangepaste regels, ontwikkeld om aan de specifieke behoeften van de organisatie te voldoen. 
De regels die nog niet bij Netskope zijn gedefinieerd en ontwikkeld op basis van een combinatie van vooraf gedefinieerde regels, eigen woordenlijsten en reguliere expressies. 
Elke gemaakte DLP-regel bevat volgende elementen:


{\small
\begin{itemize}
    \item \textbf{Regel}: De naam van de DLP-regel die wordt gebruikt.
    \item \textbf{Type}: Het type DLP-regel dat wordt gebruikt. Dit kan een DLP-regel zijn of een fingerprint.
    % \item \textbf{PAS AAN !!!!!!!!!!!!!!!!!!!!!!!!!!!!!!!!!!!!!!!!!!!!!!!!!!!!!!!!!!!}
    \item \textbf{Expressie}: De expressie die wordt gebruikt om de DLP-regel te definiëren. P staat voor \textit{predefined}, D staat voor \textit{custom dictionary} en E voor \textit{exact match}.
    \item \textbf{\textit{Vooraf gedefinieerde identificatiegegevens}}: Dit zijn entiteiten die door Netskope zijn gedefinieerd en die kunnen worden gebruikt in de DLP-regel.
    \item \textbf{\textit{Eigen dataset-gedefinieerde identificatiegegevens}}: De aangepaste entiteiten die worden gebruikt in de expressie. Hierin staan lijsten van woorden, nummers of reguliere expressies gedefinieerd. 
    \item \textbf{\textit{Regex}}: Reguliere expressies bestaan uit een reeks tekens die een patroon definiëren, zoals een rijksregister- of \gls{iban}-nummer.
    \item \textbf{\textit{Exacte overeenkomst}}: Deze overeenkomsten worden gebruikt om specifieke termen of woorden te identificeren die in de DLP-regel worden gebruikt, zoals een wachtwoord.
    \item \textbf{Gebruikte techniek}: De techniek die wordt gebruikt om de DLP-regel te definiëren, zoals reguliere expressies of vooraf gedefinieerde identificatiegegevens.
    \item \textbf{\textit{Bron(nen)}}: De bronnen die zijn gebruikt om de DLP-regel te definiëren. Dit kunnen woordenlijsten zijn, maar ook andere bronnen zoals websites of documenten.
    \item \textbf{Threshold}: De drempelwaarde die wordt gebruikt om de DLP-regel te definiëren. Dit kan \textbf{Low}, \textbf{Medium}, \textbf{High} of \textbf{Critical} zijn. Vanaf de waarde \textbf{Medium} zal Netskope de actie van de gebruiker blokkeren. \textbf{Low} is enkel een waarschuwing.
\end{itemize}
}

In tabellen~\ref{tab:custom-be-id},~\ref{tab:custom-be-iban} en~\ref{tab:custom-be-iban-terms},~\ref{tab:custom-eu-card} en~\ref{tab:custom-eu-card-terms} staan de aangepaste DLP-regels die zijn ontwikkeld voor dit onderzoek. 
Dankzij het gebruik van DLP-regels met en zonder het gebruik van \textbf{termen}, kunnen false positives sneller worden gedetecteerd. 

\begin{table}[H]
    \centering
    \small
    \scriptsize
    \begin{tabular}{p{4cm} p{10cm}}
        \toprule
        \textbf{Regel} & BE \-- National ID and Surname \\
        \midrule
        \textbf{Type} & DLP Rule \\
        \textbf{Expressie} & \texttt{(P1 OR P2) AND (P0 OR D0 OR D1)} \\
        \textbf{Vooraf gedefinieerde identificatiegegevens} & 
        (P0) \-- Surnames (International) \newline
        (P1) \-- National ID Numbers (BE) \newline
        (P2) \-- National ID Terms (BE; “RRN”) \\
        \textbf{Eigen gedefinieerde identificatiegegevens} & 
        (D0) \-- Full Names (BE) A-P \newline
        (D1) \-- Full Names (BE) P-Z \\
        \textbf{Gebruikte techniek} & Regex voor RRN in combinatie met woordenlijsten van Belgische achternamen \\
        \textbf{Bron(nen)} & \textcite{Statbel2023, Statbel2024} \\
        \textbf{Threshold} & Low: 2 \quad Medium: 10 \quad High: 25 \quad Critical: 50 \\
        % \textbf{Extra} & OBFUSCATION? https://docs.netskope.com/en/dlp-entity/#:~:text=Entities%20are%20used%20in%20a,click%20on%20the%20Entities%20tab. \\
        \bottomrule
    \end{tabular}
    \caption{Custom DLP-regel: Detectie van Belgische rijksregisternummers in combinatie met familienamen}
    \label{tab:custom-be-id}
\end{table}



% % % % % % % % \begin{table}[h]
% % % % % % % %     \centering
% % % % % % % %     \small
% % % % % % % %     \scriptsize
% % % % % % % %     \begin{tabular}{p{4cm} p{10cm}}
% % % % % % % %         \toprule
% % % % % % % %         \textbf{Regel} & BE -  \\
% % % % % % % %         \midrule
% % % % % % % %         \textbf{Type} & DLP Rule \\
% % % % % % % %         \textbf{Expressie} & \texttt{} \\
% % % % % % % %         \textbf{Vooraf gedefinieerde identificatiegegevens} & 
% % % % % % % %         (P0) - \newline
% % % % % % % %         (P1) - \newline
% % % % % % % %         (P2) -  \\
% % % % % % % %         \textbf{Eigen gedefinieerde identificatiegegevens} & 
% % % % % % % %         (D0) - \newline
% % % % % % % %         (D1) - \\
% % % % % % % %         \textbf{Gebruikte techniek} & \\
% % % % % % % %         \textbf{Bron(nen)} & \textcite{} \\
% % % % % % % %         \textbf{Threshold} & Low: 2 \quad Medium: 10 \quad High: 25 \quad Critical: 50 \\
% % % % % % % %         \bottomrule
% % % % % % % %     \end{tabular}
% % % % % % % %     \caption{Custom DLP-regel: }
% % % % % % % %     \label{tab:custom-}
% % % % % % % % \end{table}


\begin{table}[H]
    \centering
    \small
    \scriptsize
    \begin{tabular}{p{4cm} p{10cm}}
        \toprule
        \textbf{Regel} & BE \-- PCI \-- IBAN \\
        \midrule
        \textbf{Type} & DLP Rule \\
        \textbf{Expressie} & \texttt{P0 OR P1 OR P2 OR P3 } \\
        \textbf{Vooraf gedefinieerde identificatiegegevens} & 
        (P0) \-- Deposit Account Numbers (BE)
        (P1) \-- IBAN (BE; formatted)
        (P2) \-- IBAN (BE; unformatted)
        (P3) \-- IBAN (all) \\
        \textbf{Gebruikte techniek} & Vooraf gedefinieerde DLP identificatiegegevens gebruikt rond \gls{iban} en bankrekeningnummers \\
        \textbf{Bron(nen)} & Null \\
        \textbf{Threshold} & Low: 1 \quad Medium: 3 \quad High: 5 \quad Critical: 10 \\
        \bottomrule
    \end{tabular}
    \caption{Vooraf gedefinieerde DLP-regel: Detectie van Belgische \gls{iban}-nummers}
    \label{tab:custom-be-iban}
\end{table}

\begin{table}[H]
    \centering
    \small
    \scriptsize
    \begin{tabular}{p{4cm} p{10cm}}
        \toprule
        \textbf{Regel} & BE \-- PCI \-- IBAN with Bank Terms \\
        \midrule
        \textbf{Type} & DLP Rule \\
        \textbf{Expressie} & \texttt{(P2 NEAR P0) OR (P2 NEAR P1) OR (P2 NEAR P6) OR (P2 NEAR P7) OR (P2 NEAR P8) OR (P3 NEAR P0) OR (P3 NEAR P1) OR (P3 NEAR P6) OR (P3 NEAR P7) OR (P3 NEAR P8) OR (P4 NEAR P0) OR (P4 NEAR P1) OR (P4 NEAR P6) OR (P4 NEAR P7) OR (P4 NEAR P8) OR (P5 NEAR P0) OR (P5 NEAR P1) OR (P5 NEAR P6) OR (P5 NEAR P7) OR (P5 NEAR P8)} \\
        \textbf{Vooraf gedefinieerde identificatiegegevens} & 
        (P0) \-- IBAN Terms \newline 
        (P1) \-- Bank Account Terms (BE) \newline
        (P2) \-- Deposit Account Numbers (BE) \newline
        (P3) \-- IBAN (BE; formatted) \newline
        (P4) \-- IBAN (BE; unformatted) \newline
        (P5) \-- IBAN (all) \newline
        (P6) \-- Card Number Terms (English) \newline
        (P7) \-- Card Security Terms (English) \newline
        (P8) \-- Bank Account Number Terms (English) \\
        \textbf{Gebruikte techniek} & Vooraf gedefinieerde DLP identificatiegegevens gebruikt rond \gls{iban} en bankrekeningnummers in combinatie met woordenlijsten van Belgische termen. \\
        \textbf{Bron(nen)} & Null \\
        \textbf{Threshold} & Low: 1 \quad Medium: 2 \quad High: 5 \quad Critical: 9 \\
        \bottomrule
    \end{tabular}
    \caption{Vooraf gedefinieerde DLP-regel: Detectie van Belgische \gls{iban}-nummers in combinatie met banktermen}
    \label{tab:custom-be-iban-terms}
\end{table}


% \subsection{\IfLanguageName{dutch}{Persoonlijke gegevens (PII)}{Personal Data}}
% \label{subsubsec:Persoonlijke gegevens}

% \subsubsection{\IfLanguageName{dutch}{Identificatiegerichte regels}{Identification Rules}}
% \label{subsubsec:identificatiegerichte-regels}

% \subsubsection{\IfLanguageName{dutch}{Gevoelige persoonsgegevens}{Sensitive Personal Data}}
% \label{subsubsec:gevoelige-persoonsgegevens}

% \subsection{\IfLanguageName{dutch}{Betalingsgegevens (PCI)}{Financial Data}}
% \label{subsubsec:betalingsgegevens}
% % \subsubsection{\IfLanguageName{dutch}{)}{}}
% % \label{subsubsec:}
% % https://www.b2bpay.co/list-banks-europe


\section{\IfLanguageName{dutch}{Toepassingen}{Applications}}
\label{subsubsec:toepassingen}

Tabel~\ref{tab:toepassingen} geeft een overzicht van de gemaakte profielen. 
Dit zijn toepassingen van de eerder opgestelde regels in tabel~\ref{tab:datatypes_risico_uitgebreid}.

\begin{table}[H]
    \centering
    \small
    \scriptsize
    \begin{tabular}{p{3cm}p{5cm}p{6cm}}
        \toprule
        \textbf{Toepassing} & \textbf{Categorieën (Netskope)} & \textbf{Voorbeelden van applicaties} \\
        \midrule
        \textbf{SaaS}     & Collaboration, Cloud Storage, Application Suite & Slack, Teams, Teamleader, Google Drive, Jira, Confluence \\
        \textbf{Web}      & Professional Networking, Development Tools & LinkedIn, GitHub \\
        \textbf{IaaS}     & \gls{iaas}/\gls{paas}, Web Hosting, \gls{isp} & AWS, Azure, Google Cloud, OVH, DigitalOcean, Cloudflare \\
        \textbf{E-mail}    & Webmail, E-mail Outbound App & Gmail, Outlook, ProtonMail, Mailchimp, Thunderbird \\
        \textbf{Endpoint} & Endpoint DLP, Utilities & \gls{usb}, clipboard, printers, external devices \\
        \bottomrule
    \end{tabular}
    \caption{Overzicht van de verschillende toepassingen van de DLP-regels}
    \label{tab:toepassingen}
\end{table}

% In dit onderzoek wordt een Proof of Concept (PoC) klantomgeving ontwikkeld voor Evolane. 
% In deze omgeving worden zowel confidentiële als niet-confidentiële gegevens verwerkt en opgeslagen. 
% Door een Data Leakage Prevention-oplossing te implementeren, worden deze gegevens beveiligd tegen lekken. De DLP-oplossing moet voldoen aan de Belgische wetgeving, 
% waaronder de Algemene Verordening Gegevensbescherming (AVG) over persoonsgegevens (PII) en de Payment Card Industry Data Security Standards (PCI DSS) met betrekking tot betalingsgegevens (PCI). 
% De DLP-oplossing moet verder rekening houden met de NIS2-richtlijn en andere cybersecuritykaders, zoals het CCB-kader of ISO 27001. 
% De PoC bevat een Netskope-gebaseerde DLP-oplossing die confidentiële gegevens beschermt en voldoet aan de Belgische regelgeving. 

% Om een breder inzicht te verkrijgen in de implementatie en optimalisatie van Data Loss Prevention (DLP)-oplossingen, wordt in deze bachelorproef verder onderzocht hoe DLP kan worden toegepast in verschillende contexten en omgevingen. De onderstaande gebieden vormen de focus van dit onderzoek:

% \begin{itemize}
% \item \textbf{SaaS} (Software as a Service): DLP-oplossingen richten zich op gegevens in beweging (\textit{Data-in-Motion}), zoals bij het gebruik van forward proxies en bij statische gegevens (\textit{Data-at-Rest}) voor APIs.
% \item \textbf{Web}: Hier wordt DLP toegepast op geëncrypteerd verkeer en voor alle poorten, met focus op gegevens in beweging tussen systemen.
% \item \textbf{IaaS} (Infrastructure as a Service): Net zoals bij SaaS, wordt DLP toegepast op gegevens in beweging en in rust, ook 
% \item \textbf{E-mail}: Bij het beheren van e-mailverkeer, zoals webmail, richt de uitgewerkte DLP-oplossing zich zowel op gegevens in beweging als op gegevens in rust.
% \item \textbf{Endpoint}: Endpoint Data Loss Prevention (DLP) richt zich op het beschermen van gegevens in gebruik (\textit{Data-in-Use}) door middel van endpointbeveiliging. Dit houdt het volgende in: het monitoren en beperken van gegevensoverdracht via USB-opslagmedia, klemborden, printers en andere externe apparaten.
% \end{itemize}


\subsection{\IfLanguageName{dutch}{Software as a Service (SaaS)}{Software as a Service (SaaS)}}
\label{subsubsec:saas-poc}
% Forward proxy and API 
% Slack, Teams, Teamleader, Google Drive, Jira, Confluence
% \textbf{Hierin komt er een overzicht van hoe SaaS toepassingen werken met Netskope. en hoe ik deze heb geimplenteerd.}

% In de SaaS-omgeving werd de focus gelegd op het beveiligen van cloudapplicaties zoals Slack, 
% Microsoft Teams, Google Drive, Jira en Confluence. Deze applicaties vallen onder de categorieën Collaboration en Cloud Storage. 
% Door gebruik te maken van een forward proxy via de Netskope Client en API-integraties met deze diensten, 
% werd zowel Data-in-Motion als Data-at-Rest beschermd. 
% Via de API werden bestanden gescand op gevoelige informatie op het moment van upload, synchronisatie of delen. 
% Door middel van policyregels gebaseerd op GDPR- en PCI-profielen werd onder andere het delen van persoonsgegevens en creditcardinformatie geblokkeerd of gelogd.

In de \gls{saas}-omgeving worden cloudapplicaties zoals \textbf{\textit{Slack}}, \textbf{\textit{Microsoft Teams}}, \textbf{\textit{Teamleader}}, 
\textbf{\textit{Google Drive}}, \textbf{\textit{Jira}} en \textbf{\textit{Confluence}} beveiligd door een combinatie van forward proxy-verkeer en geavanceerde \gls{api}-integraties binnen Netskope. 
Deze toepassingen \ref{tab:toepassingen} vallen onder de categorieën \textit{Collaboration}, \textit{Cloud Storage} en \textit{Application Suite} \autocite{Netskope2025API}.

De \textit{\textbf{forward proxy}}-setup zorgt voor real-time controle over gebruikersverkeer, 
inclusief het scannen van bestanden en berichten tijdens het verzenden of uploaden. 
Hierdoor wordt data-in-motion onmiddellijk geanalyseerd en geblokkeerd indien nodig.

Via \gls{api}-integraties worden bestanden en berichten gescand op het moment van upload, synchronisatie of delen \autocite{Netskope2025API}. 
Zo wordt ook data-at-rest beschermd, bijvoorbeeld bij reeds opgeslagen documenten in Google Drive of bij gedeelde bestanden in Slack-kanalen \autocite{Netskope2022Slack}.

De integraties van deze applicaties met Netskope zijn reeds deels geïmplementeerd door Evolane. 
Aangezien de focus van dit onderzoek ligt op de DLP-regels, zal deze integratie niet verder uitgewerkt worden.

\subsection{\IfLanguageName{dutch}{Web}{Web}}
\label{subsubsec:web-poc}
% Forward proxy hiermee 
% LinkedIn, Github

In de webomgeving wordt uitgaand verkeer beveiligd via een forward proxy in combinatie met \gls{ssl}-inspectie. 
Netskope onderschept versleutelde \gls{https}-verzoeken en analyseert deze op de aanwezigheid van confidentiële data.
Zo wordt het dataverlies via webformulieren, berichten of uploads proactief tegengehouden. 
Deze aanpak focust op \textit{data-in-motion} en wordt toegepast op websites zoals \textbf{LinkedIn}, \textbf{GitHub}, \textbf{Messenger}, \textbf{ChatGPT} en \textbf{Google Forms}.

De eerste test om te zien of de Netskope DLP-regels correct werken, wordt uitgevoerd op de website \texttt{dlptoolbox.com}.
Het gebruikte script~\ref{lst:dlp-script} is ontworpen om testgegevens van \texttt{dlptest.com} te versturen naar het testformulier op \texttt{dlptoolbox.com}.
De testdata bevatten combinaties van persoonsnamen, en creditcardnummers (\textit{Visa} en \textit{Mastercard}), gestructureerd in \gls{csv}-formaat, zoals hieronder weergegeven:

\begin{quote}\small
\texttt{Robert Aragon, 4929-3813-3266-4295} \newline
\texttt{Ashley Borden, 5370-4638-8881-3020}
\end{quote}

Voor deze korte evaluatie wordt een \textit{Python}-script opgebouwd dat gebruikmaakt van \textbf{Selenium}. 
Het script is ontworpen om testgegevens aan de hand van batches in te voeren via het online platform \texttt{dlptoolbox.com}.
Dit platform biedt een veilige manier om DLP-regels realistisch te testen binnen de browseromgeving van de gebruiker.
Het script stuurt gegevens automatisch in batches van tien rijen naar het testformulier \ref{lst:dlp-selenium} en verzendt deze data aan de hand van een \texttt{POST}-request naar de server.
Dit wordt tijdens de uitvoering in rijen van 10 gedaan, aangezien de dataset \autocite{DLPTest2025Sample} 30 000 rijen bevat.

{\small
\begin{lstlisting}[style=custompython,caption={Versturen van batches via Selenium},label={lst:dlp-selenium}, captionpos=b, basicstyle=\small\ttfamily]
from selenium.webdriver.common.by import By
from selenium.webdriver.support import expected_conditions as EC
textarea = wait.until(EC.presence_of_element_located((By.ID, "clientDataClear")))
textarea.clear()
textarea.send_keys(batch)
\end{lstlisting}
}

Na verzending wordt de gelekte data teruggestuurd~\ref{lst:dlp-teruggelezen}. Netskope staat in alert modus, zodat iedere keer dat gevoelige data wordt gedetecteerd, een incident wordt aangemaakt. 

{\small
\begin{lstlisting}[style=custompython, caption={Verzonden data wordt teruggelezen},label={lst:dlp-teruggelezen}, captionpos=b, basicstyle=\small\ttfamily]
leaked_box = driver.find_element(By.ID, "clientDataLeaked")
leaked_data = leaked_box.get_attribute("value")
print(f"Gelekte data:\n{leaked_data}")
\end{lstlisting}
}

Elke testcyclus wordt afgesloten met een reset van het formulier om interferentie te voorkomen. 
De testgegevens en het script zijn louter \textbf{\textit{ter illustratie}} en worden niet gebruikt in de uiteindelijke evaluatie van de DLP-regels.
Het script wordt wel aangepast om aan de hand van een eigen opgestelde \textit{Ngrok}-server de data te sturen, zodat Netskope de data kon scannen~\ref{sec:automatisering-poc}.


% \begin{lstlisting}[style=custompython,label={lst:dlp-selenium},caption={Selenium-script voor batchgewijze verzending van testdata}]
%     code hier
% \end{lstlisting}

% \begin{lstlisting}[style=custompython,caption={Selenium-script voor batchgewijze verzending van testdata},label={lst:dlp-selenium}]
%     from selenium import webdriver
%     from selenium.webdriver.common.by import By
%     from selenium.webdriver.chrome.options import Options
%     from selenium.webdriver.support.ui import WebDriverWait
%     from selenium.webdriver.support import expected_conditions as EC
%     import time
    
%     # Config
%     URL = "https://dlptoolbox.com/postmaster.aspx"
%     INPUT_FILE = "dlp_testdata.csv"
%     ROWS_PER_BATCH = 10
%     WAIT = 5
    
%     # Setup Chrome
%     options = Options()
%     options.add_argument("--headless")
%     driver = webdriver.Chrome(options=options)
%     wait = WebDriverWait(driver, 10)
    
%     driver.get(URL)
%     iframe = wait.until(EC.presence_of_element_located((By.ID, "ifrPOST")))
%     driver.switch_to.frame(iframe)
    
%     with open(INPUT_FILE, 'r') as f:
%         lines = [line.strip() for line in f if line.strip()]
    
%     for i in range(0, len(lines), ROWS_PER_BATCH):
%         batch = "\n".join(lines[i:i+ROWS_PER_BATCH])
%         print(f" Versturen batch {i//ROWS_PER_BATCH + 1}:\n{batch}")
    
%         textarea = wait.until(EC.presence_of_element_located((By.ID, "clientDataClear")))
%         textarea.clear()
%         textarea.send_keys(batch)
    
%         launch_btn = wait.until(EC.element_to_be_clickable((By.ID, "btnLaunch")))
%         launch_btn.click()
%         time.sleep(WAIT)
    
%         leaked_box = driver.find_element(By.ID, "clientDataLeaked")
%         leaked_data = leaked_box.get_attribute("value")
%         print(f"Gelekte data ({len(leaked_data)} tekens):\n{leaked_data}")
%         print("—" * 50)
    
%         try:
%             reset_btn = wait.until(EC.element_to_be_clickable((By.ID, "btnReset")))
%             reset_btn.click()
%             time.sleep(1)
%         except Exception as e:
%             print(f"[!] Reset mislukt: {e}")
    
%     driver.quit()
% \end{lstlisting}
    

\subsection{\IfLanguageName{dutch}{Infrastructure as a Service (IaaS)}{Infrastructure as a Service (IaaS)}}
\label{subsubsec:iaas-poc}
% Forward proxy and API
% Linode, Akamai, AWS, Azure, Google Cloud, Digital Ocean, OVH, Vultr, Hetzner, Kinsta, Cloudflare,

In de \gls{iaas}-omgeving biedt Netskope beveiliging aan de hand van forward proxy en \gls{api}-integraties. 
Platforms zoals \textbf{\textit{AWS}}, \textbf{\textit{Azure}}, \textbf{\textit{Google Cloud}}, \textbf{\textit{DigitalOcean}}, \textbf{\textit{OVH}}, \textbf{\textit{Hetzner}}, \textbf{\textit{Linode}} 
en \textbf{\textit{Cloudflare}} vallen onder deze categorie. Deze cloudinfrastructuurplatforms worden vaak gebruikt voor het hosten van virtuele machines, opslagdiensten en netwerkcomponenten.

% De IaaS-toepassingen omvatten onder andere AWS, Azure en Google Cloud. 
% Zowel API-gebaseerde scanning van opslag (zoals S3-buckets) als realtime monitoring via proxies werden onderzocht. 
% In deze omgeving werd gecontroleerd op het onbedoeld delen van gevoelige bestanden of configuraties, 
% zoals exports van databases met persoonsgegevens. Met behulp van aangepaste DLP-profielen werd onder meer detectie uitgevoerd op Belgische identiteitsnummers en medische gegevens. 
% Incidentmeldingen werden gelogd conform de NIS2-verplichtingen.




\subsection{\IfLanguageName{dutch}{E-mail}{E-mail}}
\label{subsubsec:e-mail-poc}
% Webmail and API
% Outlook, Gmail, Exchange, Office 365, Thunderbird, Protonmail, Mailchimp, Sendinblue

Voor e-mailverkeer kan DLP worden toegepast op zowel webmail (zoals \textbf{\textit{Gmail}} en \textbf{\textit{Outlook Web Access}}) als softwareclients (zoals \textbf{\textit{Thunderbird}} en \textbf{\textit{Outlook}}).
Netskope biedt ook SMTP Proxy-ondersteuning voor het scannen van e-mailverkeer \autocite{Netskope2025Email}. Dit omvat het scannen van uitgaande e-mails op gevoelige bijlagen of tekstinhoud vóór verzending (outbound).
Deze ondersteuning ligt niet binnen de scope van dit onderzoek, maar eens deze geïmplementeerd is, werkt deze op dezelfde manier als de webmail en client-integraties.
Netskope zal voordat de e-mail wordt verzonden (\textit{data-in-motion}), de inhoud scannen op gevoelige gegevens.
Via \gls{api}-integraties met \textbf{\textit{Microsoft 365}} en \textbf{\textit{Gmail}} kunnen berichten en bijlagen worden gescand op basis van \gls{avg}-regels, \gls{pci}-gegevens en Belgische identificatiegegevens (zoals rijksregisternummer).

\subsection{\IfLanguageName{dutch}{Endpoint}{Endpoint}}
\label{subsubsec:endpoint-poc}
% Endpoint DLP
% USB, clipboard, printers, external devices

DLP kan op endpoints worden geïmplementeerd, waarbij de focus ligt op het beschermen van gegevens in gebruik (\textit{data-in-use}). 
\textcite{Netskope2025Endpoint} bekijkt en beperkt de overdracht van gevoelige gegevens via \gls{usb}-opslagmedia, klemborden, printers en andere externe apparaten.
Deze functionaliteit kan geblokkeerd of toegelaten worden op basis van de DLP-regels die zijn ingesteld.
De testen worden manueel uitgevoerd op een \textit{MacOS}-systeem, de confidentiële gegevens van reeds gedetecteerde testdata wordt opgeslagen en naar een \gls{usb}-stick gekopieerd.
In geval van een overtreding, zal Netskope een waarschuwing tonen en vervolgens de actie blokkeren.

\subsection{\IfLanguageName{dutch}{Testgegevens}{Test Data}}
\label{subsubsec:poc-testgegevens}

Voor dit onderzoek is \textbf{een achtste} van de testgegevens gebruikt van \textit{Enron's E-mail Dataset} \autocite{Cukierski2015Enron}.
Deze dataset bevat e-mails van het bedrijf Enron, zoals besproken in sectie~\ref{sec:detectienauwkeurigheid-literatuurstudie}.
De e-mails worden gecategoriseerd onder \textbf{user}~\ref{subsubsec:dataset-categorisering} en bevatten zowel confidentiële als niet-confidentiële gegevens.
Het doel van deze testgegevens is om te kijken hoeveel confidentiële gegevens gedetecteerd kunnen worden door de Netskope agent en om de detectienauwkeurigheid van de DLP-regels te evalueren.
Tijdens de evaluatie zullen de door Netskope gedetecteerde gegevens steekproefsgewijs worden vergeleken.
% Uit deze evaluatie zal een conclusie worden getrokken
% hieruit volgt een conclusie.
Een voorbeeld van zo'n e-mail is te zien in Listing~\ref{lst:dlp-e-mail-ex}.

\begin{lstlisting}[style=custompython, caption={Voorbeeld van een e-mail uit de Enron dataset}, label={lst:dlp-e-mail-ex}, captionpos=b]
Message-ID: <14280616.1075855349137.JavaMail.evans@thyme>
Date: Fri, 7 Dec 2001 09:40:02 -0800 (PST)
From: martin.cuilla@enron.com
To: rob@ramind.com
Subject: RE:
Mime-Version: 1.0
Content-Type: text/plain; charset=us-ascii
Content-Transfer-Encoding: 7bit
X-From: Cuilla, Martin </O=ENRON/OU=NA/CN=RECIPIENTS/CN=MCUILLA>
X-To: 'Rob Miller' <rob@ramind.com>
X-cc: 
X-bcc: 
X-Folder: \Martin_Cuilla_Jan2002_1\Cuilla, Martin\Sent Items
X-Origin: Cuilla-M
X-FileName: mcuilla (Non-Privileged).pst

yeah i am in the process of interviewing but i have a job at least for the next 90 days and they are working on a job offer for if i stay on past that.  unfortunately most of the other job offers are out of texas though - stamford, connecticut; columbus, ohio; and kansas city, missouri.  hopefully in a couple of weeks i will know if i am staying here or going elsewhere.

-----Original Message-----
From: Rob Miller [mailto:rob@ramind.com]
Sent: Friday, December 07, 2001 11:34 AM
To: Cuilla, Martin
Subject: 

hey marty do you still have a job?
\end{lstlisting}

Deze e-mail bevat een bijlage \texttt{X-FileName: pallen.nsf}, als het een echte e-mail was, dan zou Netskope deze bijlage scannen op confidentiële gegevens.

\section{\IfLanguageName{dutch}{Automatisering van Gegevensverwerking en Testverkeer}{Automation of Data Processing and Test Traffic}}
\label{sec:automatisering-poc}

Om de DLP-regels te evalueren wordt een testscript ontwikkelt in \textbf{Python} dat gebruikersinteracties simuleert via een publieke webinterface. 
Deze interface wordt gehost met behulp van een lokale Flask-applicatie die tijdelijk toegankelijk is gemaakt via ngrok \autocite{Ngrok2025Flask}. 
Een Selenium WebDriver bestuurt vervolgens een browser om gebruikersgedrag te simuleren.
De structuur van het script is onderverdeeld in onderstaande subsecties, die elk een specifiek aspect van de automatisering en evaluatie van DLP-regels behandelen.
% De structuur van het script is gebaseerd op vier belangrijke pijlers:

% \begin{itemize}[label=\textbullet, leftmargin=2em, itemsep=0.5em]
%     \item \textbf{Ngrok Webinterface}~\ref{subsubsec:ngrok-webinterface}
%     \item \textbf{Dataset categorisering}~ \ref{subsubsec:dataset-categorisering}
%     \item \textbf{Browserautomatisering}~\ref{subsubsec:browserautomatisering}
%     \item \textbf{Clipboard met Pyperclip}~\ref{subsubsec:clipboard}
%     \item \textbf{Gebruikersinteractie via ActionChains}~\ref{subsubsec:gebruikersinteractie}
%     \item \textbf{Logging en herstartbaarheid}~\ref{subsubsec:logging-herstartbaarheid}
% \end{itemize}

\subsection{\IfLanguageName{dutch}{Ngrok Webinterface}{Ngrok Web Interface}}
\label{subsubsec:ngrok-webinterface}

Voor het opzetten van een tijdelijke webinterface wordt gebruik gemaakt van \textbf{ngrok} \autocite{Ngrok2025Flask}.
Deze tool maakt gratis lokale webapplicaties tijdelijk toegankelijk via een gehashte \gls{url} \texttt{https://<hash>.ngrok-free.app}.
% \autocite{Ngrok2025NAT} ts 
\ref{lst:ngrok-webinterface}.
% Ngrok biedt een veilige tunnel naar de lokale Flask-webapplicatie die draait op poort 5000 \autocite{Ngrok2025NAT}~\ref{lst:ngrok-webinterface}.

\begin{lstlisting}[style=custompython, label={lst:ngrok-webinterface}, caption={Ngrok configuratie voor Flask webinterface}, captionpos=b]
from flask import Flask, request, render_template

app = Flask(__name__)
@app.route('/', methods=['GET', 'POST'])
def index():
    user_input = None
    if request.method == 'POST':
        user_input = request.form.get('sensitive_data')
        print(f"[RECEIVED DATA]: {user_input}")
    return render_template('form.html', user_input=user_input)

if __name__ == '__main__':
    app.run(port=5000)
\end{lstlisting}

\subsection{\IfLanguageName{dutch}{Dataset categorisering}{Dataset Categorization}}
\label{subsubsec:dataset-categorisering}

Aangezien Enron's e-mail-dataset \textbf{517 401} verschillende e-mails bevat \autocite{Cukierski2015Enron}, wordt de dataset gecategoriseerd per \textbf{user} \ref{lst:e-mail-summary}. 
Elke e-mail bevat informatie over de \textbf{user} en wat voor \textbf{type/token} e-mail het is. Een voorbeeld hiervan is te zien in de \gls{csv}-file \ref{lst:dlp-e-mail-ex}.
De categorisering gebeurt aan de hand van een \gls{csv}-bestand dat \textbf{file}, \textbf{tokens} en \textbf{char\_len} als kolommen heeft.
Dit is een samenvattend bestand om later bij de evaluatie van Netskope incidenten~\ref{sec:incidenten-resultaten} de e-mails te kunnen terugvinden. 

\begin{lstlisting}[style=custompython, label={lst:dlp-csv}, caption={emails\_summary.csv \gls{csv}-bestand}, captionpos=b]
file,tokens,char_len
allen-p/_sent_mail/1.,187,492
allen-p/_sent_mail/10.,350,1276
\end{lstlisting}

\begin{lstlisting}[style=custompython, label={lst:dlp-categorisering}, caption={Categorisering van de dataset per gebruiker}, captionpos=b]
    for row in tqdm(data.iter_rows(), total=data.height, desc="Verwerken"):
        file, msg = row
        if msg is None:
            continue
        token_count = count_tokens(msg)
        char_len = len(msg)
        writer.writerow([file, token_count, char_len])
\end{lstlisting}


\subsection{\IfLanguageName{dutch}{Browserautomatisering}{Browser Automation}}
\label{subsubsec:browserautomatisering}

Het script maakt gebruik van Selenium WebDriver om een browser te automatiseren.
Deze WebDriver kan worden geconfigureerd om een browser te besturen en gebruikersinteracties te simuleren.

\begin{lstlisting}[style=custompython, label={lst:dlp-chrome}, caption={Selenium-script voor browserautomatisering}, captionpos=b]
    driver = webdriver.Chrome(options=options)
\end{lstlisting}

Aangezien de evaluatie gebaseerd is op realistische testdata, moet het gedrag van de browser ook realistisch zijn.
Volgende \textcite{Selenium2025WebDriver} best practices worden toegepast om de browser te laten functioneren als een echte gebruiker:

\begin{lstlisting}[style=custompython,label={lst:dlp-stealth},caption={Best practice voor browserautomatisering}, captionpos=b]
    driver.execute_cdp_cmd('Page.addScriptToEvaluateOnNewDocument', {
        'source': "Object.defineProperty(navigator, 'webdriver', {
            get: () => undefined });"
    })
\end{lstlisting}

\subsection{\IfLanguageName{dutch}{Clipboard met Pyperclip}{Clipboard with Pyperclip}}
\label{subsubsec:clipboard}

Om de e-mail-dataset in het formulier te injecteren, wordt pyperclip gebruikt~\ref{lst:dlp-pyperclip}.
Deze \textit{Python} module maakt het mogelijk om tekst te kopiëren naar het klembord en het te plaatsen in het formulier.
% Deze \textit{Python} module maakt het mogelijk om tekst naar het klembord te kopiëren en van het klembord te plakken \autocite{Pyperclip2025}.

\begin{lstlisting}[style=custompython, label={lst:dlp-pyperclip}, caption={Clipboardmanipulatie met Pyperclip}, captionpos=b]
    pyperclip.copy(content)
\end{lstlisting}

\subsection{\IfLanguageName{dutch}{Gebruikersinteractie via ActionChains}{User Interaction via ActionChains}}
\label{subsubsec:gebruikersinteractie}

De Netskope Client is geïnstalleerd op een \textit{MacOS} machine. De interactie met het formulier gebeurde eerste met de standaard Selenium-methode \texttt{\detokenize{textarea.send_keys(..)}}. 
Deze methode voert karakters individueel in in het tekstveld en is niet altijd even betrouwbaar, vooral niet bij het invoeren van e-mails boven de \textbf{10 000} karakters oftewel \textbf{10 KB}.
De interactie met het formulier gebeurt hierdoor met \textbf{ActionChains} \autocite{Selenium2025ActionChains}~\ref{lst:dlp-actionchain}. 
Dit laat toe om toetsenbordcommando's zoals \texttt{Cmd+V} te simuleren, wat de invoer van grote hoeveelheden tekst vergemakkelijkt en versnelt.

\begin{lstlisting}[style=custompython,label={lst:dlp-actionchain},caption={gebruikersinteractie via ActionChains}, captionpos=b]
    ActionChains(driver) \
    .key_down(Keys.COMMAND) \
    .send_keys('v') \
    .key_up(Keys.COMMAND) \
    .perform()
\end{lstlisting}

Zoals toegelicht in de \textcite{Selenium2025ActionChains} ActionChains-documentatie maakt deze aanpak het mogelijk om gebruikersacties zoals hover, drag-and-drop en key-combinaties uit te voeren.

\subsection{\IfLanguageName{dutch}{Logging en herstartbaarheid}{Logging and Restartability}}
\label{subsubsec:logging-herstartbaarheid}

Bij elke succesvolle verzending registreert het script het tijdstip (\textbf{timestamp}), 
de bestandssleutel (\textbf{user} en \textbf{token}) en de grootte (\textbf{char\_len}) in een logbestand \ref{lst:dlp-logging}.

\begin{lstlisting}[style=custompython,label={lst:dlp-logging}, caption={Logging van verzonden e-mails}, captionpos=b]
def log_email_send(file_key, content):
    ts = datetime.now().strftime("%-m/%-d/%y %-I:%M %p")
    size = len(content.encode('utf-8'))
    header = not os.path.exists(LOG_FILE)
    with open(LOG_FILE, "a", newline="", encoding="utf-8") as f:
        w = csv.writer(f)
        if header:
            w.writerow(["timestamp","file","size_bytes"])
        w.writerow([ts, file_key, size])
\end{lstlisting}

Het script logt de laatst verzonden e-mail in een apart tekstbestand, zodat het bij een herstart voorkomt dat reeds verzonden e-mails opnieuw worden verzonden \ref{lst:dlp-last-sent}.

\begin{lstlisting}[style=custompython,label={lst:dlp-last-sent}, caption={Logging van laatst verzonden bericht}, captionpos=b]
with open(LAST_SENT_FILE, "w") as f:
    f.write(fk)
\end{lstlisting}

\section{\IfLanguageName{dutch}{Netskope alerts}{Netskope alerts}}
\label{subsubsec:netskope-alerts}

Tijdens de evaluatieperiode~\ref{sec:evaluatie-criteria} worden de DLP-profielen in \textbf{\texttt{Alert}} of \textbf{\texttt{User Alert}} modus gebruikt.
In de \textbf{\texttt{User Alert}} modus worden overtredingen onderschept en worden gebruikers gewaarschuwd.
Zo een waarschuwing, zoals te zien in figuur~\ref{fig:netskope}, zal de gebruiker de optie geven om de actie te annuleren of door te zetten.
Dit geeft de gebruiker de kans om te leren of \textit{\textbf{False Positives}} te rapporteren.


\begin{figure}[H]
    \centering
    \includegraphics[width=0.8\textwidth]{img/netskopealert.png}
    \caption{Netskope \textbf{\texttt{User Alert}} voor een DLP-regel}
    \label{fig:netskope}
\end{figure}

Alle incidenten die worden gegenereerd door Netskope worden opgeslagen in de Netskope Cloud en kunnen worden bekeken via de Netskope webinterface~\ref{fig:netskope_incidenten}.

\begin{figure}[H]
    \centering
    \scriptsize
    \includegraphics[width=0.99\textwidth]{img/netskope_incidents2.png}
    \caption{Incidenten in Netskope}
    \label{fig:netskope_incidenten}
\end{figure}
