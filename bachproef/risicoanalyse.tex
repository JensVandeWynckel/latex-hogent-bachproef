%%=============================================================================
%% Risicoanalyse
%%=============================================================================

\chapter{\IfLanguageName{dutch}{Risicoanalyse}{Risk Analysis}}
\label{ch:risicoanalyse}

\section{\IfLanguageName{dutch}{Inleiding}{Introduction}}
\label{sec:inleiding_risicoanalyse}

Deze risicoanalyse evalueert hoe goed regex-gebaseerde detectieregels functioneren bij het identificeren van gevoelige gegevens. 
Hierbij wordt gekeken naar de waarschijnlijkheid van datalekken op basis van datatypes en hoe Netskope probabiliteit toepast in DLP-detectie. 
Deze probabiliteit kan aangepast worden om de detectieprestaties te verbeteren. Voor sommige datatypes is de kans op een datalek hoger dan voor andere. 
Hierbij zouden false positives en false negatives vermeden moeten worden.

De beoordeling wordt uitgevoerd met behulp van:
\begin{itemize}
    \item Een waarschijnlijkheid van gevoelige data-combinaties (bijv.\ naam + bankrekeningnummer).
    \item Een vergelijking met Netskope's probabilistische detectiemethoden.
    \item Een confusion matrix om de prestaties van de regex-detectieregels te evalueren.
\end{itemize}


% De risicoanalyse voor DLP (Data Loss Prevention) sluit aan bij de methodologie zoals beschreven in NIST SP 800-30 Rev. 1 en ISO 27005. 
% Volgens Fikri et al. (2019) bestaat een effectieve risicoanalyse uit een combinatie van impact- en waarschijnlijkheidsbeoordelingen, 
% waarbij elk risico wordt gekwantificeerd op basis van de ernst van de gevolgen en de waarschijnlijkheid van optreden. 
% Hierbij wordt rekening gehouden met factoren zoals vertrouwelijkheid, integriteit en beschikbaarheid van de gegevens.

% De tabel in deze studie categoriseert gegevensrisico’s op basis van een vijfpuntsschaal, 
% vergelijkbaar met de impact- en waarschijnlijkheidstabellen in NIST SP 800-30. 
% Net zoals in de methode van Fikri et al. (2019) wordt in deze analyse onderscheid gemaakt tussen:

% Impactniveau: De ernst van de schade die een datalek kan veroorzaken, variërend van "zeer laag" (minimale impact) tot "zeer hoog" (catastrofale gevolgen).
% Waarschijnlijkheidsniveau: De kans dat een bedreiging zich manifesteert, gebaseerd op historische data en systeemzwakheden.
% Een belangrijk verschil tussen deze aanpak en de methode uit NIST SP 800-30 is de specifieke focus op datatypes en hun combinaties. 
% Terwijl de methode van Fikri et al. (2019) zich richt op algemene risico’s binnen informatiesystemen, 
% hanteert deze studie een meer gedetailleerde indeling voor PII (Persoonlijk Identificeerbare Informatie), 
% authenticatiegegevens en bedrijfsgevoelige data, wat resulteert in een praktisch toepasbare DLP-risicobeoordeling.

% Door deze principes toe te passen, wordt een semi-kwantitatieve methode gehanteerd die niet alleen de potentiële impact van een datalek in kaart brengt, 
% maar ook de waarschijnlijkheid van misbruik door kwaadwillenden of interne fouten. 
% Dit maakt de risicoanalyse zowel consistent met NIST SP 800-30 als toepasbaar voor specifieke DLP-beleidscategorieën.

\section{\IfLanguageName{dutch}{Data en risicobeoordeling}{Data and Risk Assessment}}
\label{sec:data_risico}

\subsubsection{Waarschijnlijkheid van gevoelige gegevens}
Niet elk individueel gegevenselement vormt een even groot risico. Een e-mailadres op zich is minder gevoelig dan een naam gekoppeld aan een bankrekeningnummer. 
% De volgende tabel toont de gewichten toegekend aan verschillende datatypes en combinaties:

% In tabel \ref{tab:datatypes_risico} worden de waarschijnlijkheidsgewichten van verschillende datatypes en combinaties getoond. \textcite{al2019risk}
% De waarden variëren van 1 tot 5 en geven als volgt de risicograad aan:

Volgens het model van \textcite{AlFikri2019} vormen impact en waarschijnlijkheid de basis voor risico\-inschattingen. 
NIST SP 800-30 Rev. 1 \autocite{NIST800-30} en ISO 27005 \autocite{IOSIEC2022} vormen de basis van dit model.
Het model kwantificeert risico's met een schaal van 0 tot 10. 
Tabel \ref{tab:datatypes_risico} toont de waarschijnlijkheidsgewichten van verschillende datatypes en combinaties, 
met bijbehorende impactscores op een schaal van 1 tot 5.
Deze schaal is beter geschikt voor de specifieke context van DLP-risico's bij Evolane.
Tabel \ref{tab:risico_thresholds} definieert de thresholds op basis van de gehanteerde schaal.

\begin{itemize}
    \item \textbf{1: Zeer Laag Risico} Gegevens die op zichzelf weinig impact hebben bij een lek. Het verlies vormt geen directe bedreiging voor privacy of veiligheid, zoals een losse naam of publiek IP-adres.
    \item \textbf{2: Laag Risico} Individuele gegevens met enige waarde voor aanvallers, zoals een openbaar e-mailadres zonder aanvullende context.
    \item \textbf{3: Gemiddeld Risico} Combinaties van persoonsgegevens die identiteitsdiefstal kunnen vergemakkelijken, zoals naam + adres of naam + geboortedatum.
    \item \textbf{4: Hoog Risico} Gevoelige informatie, zoals financiële gegevens, die een aanzienlijke impact kunnen hebben bij een lek, zoals bankrekeningnummers of vertrouwelijke bedrijfsinformatie.
    \item \textbf{5: Zeer Hoog Risico} Kritieke gegevenscombinaties die grootschalige beveiligingsrisico's kunnen veroorzaken of ernstige schade kunnen aanrichten, zoals toegangscertificaten of gevoelige financiële informatie.
\end{itemize}
% \begin{table}[h]
%     \centering
%     \small
%     \begin{tabular}{|l|c|}
%     \hline
%     \textbf{Datatype/combinatie} & \textbf{Risicogewicht (1-5)} \\
%     \hline
%     \multicolumn{2}{|l|}{\textbf{Persoonlijke Identificeerbare Informatie (PII)}} \\
%     \hline
%     Naam alleen                 & 1 \\
%     Publiek IP-adres            & 1 \\
%     Naam + E-mailadres          & 2 \\
%     Naam + Telefoonnummer       & 2 \\
%     Naam + Geboortedatum        & 3 \\
%     Naam + BSN (NL) / NISS (BE) & 4 \\
%     Paspoortnummer              & 4 \\
%     Privaat IP-adres (intern)   & 2 \\
%     \hline
%     \multicolumn{2}{|l|}{\textbf{Betaalkaart \& PCI Gegevens}} \\
%     \hline
%     Bankrekeningnummer alleen   & 2 \\
%     Naam + Bankrekeningnummer   & 4 \\
%     Creditcardnummer            & 4 \\
%     Naam + Creditcardnummer     & 5 \\
%     CVV/CVC-code alleen         & 1 \\
%     CVV/CVC-code + Creditcardnummer & 5 \\
%     PIN-code + Kaartnummer      & 5 \\
%     \hline
%     \multicolumn{2}{|l|}{\textbf{Authenticatie \& Inloggegevens}} \\
%     \hline
%     E-mailadres alleen          & 1 \\
%     E-mailadres + Wachtwoord    & 5 \\
%     API-sleutel                 & 5 \\
%     JWT-token                   & 4 \\
%     SSH-private key             & 5 \\
%     \hline
%     \multicolumn{2}{|l|}{\textbf{Bedrijfsgevoelige Data}} \\
%     \hline
%     Klantnaam + Contractdetails & 4 \\
%     Klantnaam + Factuurgegevens & 5 \\
%     Interne projectinformatie   & 3 \\
%     Gevoelige e-mails (bijv. NDA's) & 4 \\
%     Configuratiebestanden met credentials & 5 \\
%     \hline
%     \end{tabular}
%     \caption{Waarschijnlijkheid van datatypes voor DLP-risico}
%     \label{tab:datatypes_risico}
% \end{table}


\begin{table}[h]
    \centering
    \small
    \scriptsize
    \begin{tabular}{|c|c|c|c|c|}
    \hline
    \textbf{Datatype/combinatie} & \textbf{Waarschijnlijkheid (1-5)} & \textbf{Impact (1-5)} & \textbf{Totaal Risico} & \textbf{Categorie} \\
    \hline
    \multicolumn{5}{|l|}{\textbf{Persoonlijke Identificeerbare Informatie (PII)}} \\
    \hline
    Naam alleen                 & 5 & 1 & 5  & Laag \\
    Publiek IP-adres            & 4 & 1 & 4  & Laag \\
    Naam + E-mailadres          & 5 & 3 & 15 & Hoog \\
    Naam + Telefoonnummer       & 4 & 3 & 12 & Gemiddeld \\
    Naam + Geboortedatum        & 4 & 4 & 16 & Hoog \\
    Naam + BSN (NL) / NISS (BE) & 3 & 5 & 15 & Hoog \\
    Paspoortnummer              & 2 & 5 & 10 & Gemiddeld \\
    Privaat IP-adres (intern)   & 3 & 2 & 6  & Laag \\
    \hline
    \multicolumn{5}{|l|}{\textbf{Betaalkaart \& PCI Gegevens}} \\
    \hline
    Bankrekeningnummer alleen   & 3 & 2 & 6  & Laag \\
    Naam + Bankrekeningnummer   & 4 & 4 & 16 & Hoog \\
    Creditcardnummer            & 3 & 4 & 12 & Gemiddeld \\
    Naam + Creditcardnummer     & 3 & 5 & 15 & Hoog \\
    CVV/CVC-code alleen         & 2 & 3 & 6  & Laag \\
    CVV + Creditcardnummer      & 3 & 5 & 15 & Hoog \\
    PIN-code + Kaartnummer      & 2 & 5 & 10 & Gemiddeld \\
    \hline
    \multicolumn{5}{|l|}{\textbf{Authenticatie \& Inloggegevens}} \\
    \hline
    E-mailadres alleen          & 5 & 1 & 5  & Laag \\
    E-mailadres + Wachtwoord    & 5 & 5 & 25 & Zeer hoog \\
    API-sleutel                 & 4 & 5 & 20 & Zeer hoog \\
    JWT-token                   & 3 & 4 & 12 & Gemiddeld \\
    SSH-private key             & 3 & 5 & 15 & Hoog \\
    \hline
    \multicolumn{5}{|l|}{\textbf{Bedrijfsgevoelige Data}} \\
    \hline
    Klantnaam + Contractdetails & 4 & 4 & 16 & Hoog \\
    Klantnaam + Factuurgegevens & 4 & 5 & 20 & Zeer hoog \\
    Interne projectinformatie   & 4 & 3 & 12 & Gemiddeld \\
    Gevoelige e-mails (bv. NDA's) & 3 & 4 & 12 & Gemiddeld \\
    Config-bestanden met credentials & 3 & 5 & 15 & Hoog \\
    \hline
    \end{tabular}
    \caption{Risico-analyse van datatypes voor DLP-configuratie binnen een Sales/IT-omgeving}
    \label{tab:datatypes_risico_uitgebreid}
\end{table}



\textbf{Toelichting bij belangrijke combinaties:}
\begin{enumerate}
    \item \textbf{Naam + geboortedatum + IBAN:} Zeer gevoelig voor identiteitsfraude; valt onder \gls{avg}.
    \item \textbf{E-mail + wachtwoord:} Hoog risico op account takeovers; vaak doelwit bij datalekken.
    \item \textbf{API-sleutels en config-bestanden:} Vaak hardcoded in code, eenvoudig te lekken via cloudstorage.
    \item \textbf{Creditcardgegevens:} Vallen onder \gls{pcidss}; verwerking en opslag zijn streng gereguleerd.
    \item \textbf{Biometrie en gezondheid:} Speciale categorie onder \gls{avg}; impact bij lek is zeer hoog.
    \item \textbf{Tokens (JWT/OAuth):} Directe toegang tot accounts/sessies bij lek; iets lagere waarschijnlijkheid.
\end{enumerate}






\begin{table}[h]
    \centering
    \small
    \begin{tabular}{l c c c c}
        \toprule
        \textbf{Impact Level} & \textbf{Low} & \textbf{Medium} & \textbf{High} & \textbf{Critical} \\
        \midrule
        % 1 - Zeer Laag     & $\leq$ 4   & 9   & 14  & $\geq$ 23 \\
        1 - Zeer Laag     &  4   & 9   & 14  &  23 \\
        2 - Laag          &  2   & 4   & 9   &  14 \\
        3 - Gemiddeld     &  1   & 2   & 4   &  9 \\
        4 - Hoog          &  1   & 2   & 4   &  9 \\
        5 - Zeer Hoog     &  1   & 2   & 3   &  5 \\
        \bottomrule
    \end{tabular}
    \caption{Threshold per impactniveau voor risico-inschatting}
    \label{tab:risico_thresholds}
\end{table}


\begin{table}[h]
    \centering
    \small
    \begin{tabular}{l l l}
        \toprule
        \textbf{Impact Level} & \textbf{Severity} & \textbf{Action Taken} \\
        \midrule
        1-3 (Zeer Laag-Gemiddeld) & Low      & Alert only \\
                                  & Medium   & Block \\
                                  & High     & Block \\
                                  & Critical & Block and escalate \\
        4-5 (Hoog-Zeer Hoog)      & Low      & Block \\
                                  & Medium   & Block \\
                                  & High     & Block and escalate \\
                                  & Critical & Block, escalate \\
        \bottomrule
    \end{tabular}
    \caption{Actie per combinatie van impactniveau en risicoseverity}
    \label{tab:risico_acties}
\end{table}



\textbf{Hier komt nog een MINDMAP Om te tonen hoe alle soorten datatypen met elkaar verbonden zijn.}

\subsection {\IfLanguageName{dutch}{NIS2-richtlijn}{NIS2 Directive}}
\label{sec:nis2_richtlijn}

\textbf{In deze secties komen nog de verschillende aspecten van de NIS2-richtlijn aan bod. Deze worden hier opgelijst en hier zal ook komen waar in het DLP model dit aan bod komt.}

% Een risicobeoordeling rond de NIS2-richtlijn bevat volgende elementen:

% \begin{itemize}
%     \item \textbf{Identificatie van netwerken en informatiesystemen} 
%     \item \textbf{Risicoanalyse} van deze netwerken en systemen, inclusief kwetsbaarheden en bedreigingen.
%     \item \textbf{Beveiligingsmaatregelen} die zijn genomen om risico's te beperken, bijvoorbeeld thresholds, encryptie en toegangscontroles.
%     \item \textbf{Incidentenbehandeling} voor het omgaan met beveiligingsincidenten. 
% \end{itemize}

% Evolane gebruikt cloudgebaseerde oplossingen voor het hosten van netwerken en informatiesystemen, wat betekent dat de netwerken en informatiesystemen van de organisatie zich extern bevinden.
% Voor DLP om

% De gehele identificatie van netwerken en informatiesystemen is al reeds gebeurd. Dit onderzoek zal zich vooral richten op de risicoanalyse en de beveiligingsmaatregelen.
% Aangezien dit onderzoek gericht is op de implementatie van DLP-regels binnen Netskope, is de focus van de risicoanalyse vooral gericht op de dataclassificatie en de impact van datalekken. 
% Beveiligingsmaatregelen zoals encryptie en toegangscontroles zijn al in gebruik binnen Netskope. De thresholds zijn terug te vinden in tabellen \ref{tab:risico_thresholds} en \ref{tab:risico_acties}.
% Incidentenbehandeling wordt besproken in Hoofdstuk \ref{ch:resultaten}, aangezien dit pas kan na de initiele implementatie en evaluatie van de DLP-regels.


% Evolane maakt gebruik van cloudgebaseerde oplossingen, wat betekent dat de netwerken en informatiesystemen zich buiten de organisatie bevinden. 
% De identificatie van deze systemen is al afgerond. Dit onderzoek concentreert zich daarom op de risicoanalyse en de evaluatie van bestaande en nieuwe beveiligingsmaatregelen.

% Omdat dit onderzoek specifiek focust op de implementatie van DLP-regels binnen Netskope, 
% ligt de nadruk binnen de risicoanalyse op dataclassificatie en de mogelijke gevolgen van datalekken. 
% Netskope past momenteel al beveiligingsmaatregelen toe zoals encryptie en toegangsbeheer, 
% wat beschreven staat in hun Security Cloud-documentatie\autocite{NetskopeSecurityCloud}. 
% De risicodrempels en bijhorende actiedrempels zijn opgenomen in tabellen \ref{tab:risico_thresholds} en \ref{tab:risico_acties}.

% De afhandeling van beveiligingsincidenten wordt verder behandeld in hoofdstuk \ref{ch:resultaten}, na de implementatie en evaluatie van de DLP-regels.



\subsection {\IfLanguageName{dutch}{ISO 27001}{ISO 27001}}
\label{sec:iso_27001}
% Vermelden  waar DLP vereist is in ISO27001, namelijk controls Annex A - A.5.36 Data leakage prevention , maar ook relevante controls A.5.12 Classification of InformationA.5.13 Labeling of Information


\subsection {\IfLanguageName{dutch}{AVG}{GDPR}}
\label{sec:avg}

\subsection {\IfLanguageName{dutch}{PCI DSS}{PCI DSS}}
\label{sec:pci_dss}

\subsection {\IfLanguageName{dutch}{DORA}{DORA}} % Nog toe te voegen aan literatuurstudie
\label{sec:dora}

% STUDIE VOOR IEDEREEN https://chatgpt.com/c/6829e2e1-262c-8010-8b7a-a362d5f19a38 !!!!!!!!!!!!!!!!!!!!!!!!!!!!!!!!!!!!!!!!!!!!!!!!!!!!!!!!!!!!!!!!!!!!!!!!!!!!!!!!!!!!!!!!!

\subsection{\IfLanguageName{dutch}{CCB Framework}{CCB Framework}}
\label{sec:ccb_framework}

\subsection{\IfLanguageName{dutch}{EU Cybersecurity Act}{EU Cybersecurity Act}}
\label{sec:eu_cybersecurity_act}

\subsection{\IfLanguageName{dutch}{Schrems II}{Schrems II}}
\label{sec:schrems_ii}

% \autocite{nis2directive} 

% https://www.chatpdf.com/nl/c/ErJi6REPHkB5hrLeGejEK



% \section{Probabilistische detectie in Netskope}
% Netskope gebruikt een combinatie van regex-detectie en machine learning om waarschijnlijkheidsscores toe te kennen aan mogelijke datalekken \autocite{Netskope2023}. Dit werkt als volgt:

% \begin{itemize}
%     \item Pattern matching met regex geeft een score op basis van het aantal matches.
%     \item Contextuele analyse houdt rekening met omliggende woorden om false positives te reduceren.
%     \item Confidence levels van Netskope geven een confidence score toe op basis van detectiepatronen (bijv. een bankrekeningnummer dat altijd 16 cijfers heeft versus een willekeurige numerieke string).
% \end{itemize}

% De waarschijnlijkheid per detectie wordt gemodelleerd als volgt:

% \begin{equation}
%     P(\text{Datalek}) = P(\text{Match}) \times P(\text{Context}) \times P(\text{Verificatie})
% \end{equation}

% waar:
% \begin{itemize}
%     \item \( P(\text{Match}) \) Kans dat de regex correct een patroon herkent.
%     \item \( P(\text{Context}) \) Kans dat omliggende context de match versterkt (bijv. het woord "rekeningnummer" naast een reeks cijfers).
%     \item \( P(\text{Verificatie}) \) Kans dat de detectie correct wordt geclassificeerd als gevoelige data.
% \end{itemize}

\section{\IfLanguageName{dutch}{Evaluatie van detectieprestaties}{Evaluation of Detection Performance}}
\label{sec:evaluatie_detectie}

\subsection{Confusion Matrix voor regex-detectie}
Om de effectiviteit van de regex-gebaseerde detectie te beoordelen, gebruiken we een confusion matrix. Dit laat zien hoe vaak een detectie correct of foutief is:

\textbf{Hier ben ik nog niet tevreden mee. Ik zou graag nog alle tabellen uitleggen en de formules tijdens de poc implementeren.}

\begin{table}[h]
    \centering
    \begin{tabular}{|c|c|c|}
        \hline
        \textbf{} & \textbf{Werkelijke Gevoelige Data} & \textbf{Geen Gevoelige Data} \\ \hline
        \textbf{Gedetecteerd als gevoelig} & True Positives (TP) & False Positives (FP) \\ \hline
        \textbf{Niet gedetecteerd als gevoelig} & False Negatives (FN) & True Negatives (TN) \\ \hline
    \end{tabular}
    \caption{Confusion Matrix voor regex-detectie}
    \label{tab:confusion_matrix}
\end{table}

De belangrijkste prestatiemetrics zijn:
\begin{itemize}
    \item Precision \( = \frac{TP}{TP + FP} \) Hoe vaak een detectie correct was.
    \item Recall \( = \frac{TP}{TP + FN} \) Hoeveel gevoelige data correct werd gedetecteerd.
    \item F1-score \( = 2 \times \frac{\text{Precision} \times \text{Recall}}{\text{Precision} + \text{Recall}} \) Gebalanceerde prestatiemeting.
\end{itemize}



% --------------------------------------------------------------------------------------------------------------------------------------------------------------------------------------
% Specificiteit, positief voorspellende waarde, negatief voorspellende waarde, likelihood ratio's, enz. kunnen ook worden berekend om de prestaties van de detectie te evalueren \autocite{Fawcett2006}.
% Sensitivity, specificity, predictive values, and likelihood ratios \autocite{Fawcett2006}.
% PPV betekent positief voorspellende waarde, hiermee wordt bedoeld hoeveel van de positieve voorspellingen daadwerkelijk positief zijn. 
% NPV betekent negatief voorspellende waarde, hiermee wordt bedoeld hoeveel van de negatieve voorspellingen daadwerkelijk negatief zijn..
% https://www.researchgate.net/publication/49650721_Sensitivity_specificity_predictive_values_and_likelihood_ratios/figures?lo=1 
% --------------------------------------------------------------------------------------------------------------------------------------------------------------------------------------



% \section{\IfLanguageName{dutch}{Resultaten en Conclusie}{Results and Conclusion}}
% \label{sec:resultaten_conclusie}

% \subsection{Resultaten van regex-evaluatie}
% Na het uitvoeren van de regex-tests op verschillende datasets, zijn de volgende resultaten behaald:

% \begin{itemize}
%     \item \textbf{Precision:} 85\% De regex-detectie leverde relatief weinig vals positieven op.
%     \item \textbf{Recall:} 72\% Een deel van de gevoelige data werd niet correct gedetecteerd.
%     \item \textbf{F1-score:} 78\% De balans tussen precision en recall toont ruimte voor verbetering.
% \end{itemize}

% \subsection{Verbeterpunten voor detectie}
% Op basis van de confusion matrix en prestatiemetrics zijn er verbeteringen mogelijk:
% \begin{itemize}
%     \item Optimalisatie van regex-patterns om vals negatieven te verminderen.
%     \item Betere contextdetectie binnen Netskope om vals positieven te verlagen.
%     \item Gebruik van machine learning-ondersteuning voor complexere datacombinaties.
% \end{itemize}

% \subsection{Conclusie}
% De evaluatie laat zien dat regex-gebaseerde detectie een effectieve methode is voor DLP, maar dat probabilistische modellen zoals die van Netskope een belangrijk voordeel bieden in contextuele en adaptieve detectie. Toekomstig werk kan zich richten op het verbeteren van recall door geavanceerde patroonherkenning en contextgebaseerde AI-modellen te gebruiken.







% \chapter{\IfLanguageName{dutch}{Risicoanalyse}{Risk Analysis}}%
% \label{ch:risicoanalyse}

% % https://www.scribbr.nl/modellen/risicoanalyse/

% % https://www.cyberpilot.io/cyberpilot-blog/what-a-risk-analysis-is-and-a-step-by-step-guide-to-guide-to-do-one-free-template?utm_source=chatgpt.com

% % https://docs.trendmicro.com/en-us/documentation/article/trend-micro-email-security-online-help-dlp-compliance-templ?utm_source=chatgpt.com

