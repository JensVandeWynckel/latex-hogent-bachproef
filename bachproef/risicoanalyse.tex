%%=============================================================================
%% Risicoanalyse
%%=============================================================================

\chapter{\IfLanguageName{dutch}{Risicoanalyse}{Risk Analysis}}
\label{ch:risicoanalyse}

\section{\IfLanguageName{dutch}{Inleiding}{Introduction}}
\label{sec:inleiding_risicoanalyse}

Deze risicoanalyse evalueert hoe goed regex-gebaseerde detectieregels functioneren bij het identificeren van gevoelige gegevens. 
Hierbij wordt gekeken naar de waarschijnlijkheid van datalekken op basis van datatypes en hoe Netskope probabiliteit toepast in DLP-detectie. 
Deze probabiliteit kan aangepast worden om de detectieprestaties te verbeteren. Voor sommige datatypes is de kans op een datalek hoger dan voor andere. 
Hierbij zouden false positives en false negatives vermeden moeten worden.

De beoordeling wordt uitgevoerd met behulp van:
\begin{itemize}
    \item Een waarschijnlijkheid van gevoelige data-combinaties (bijv.\ naam + bankrekeningnummer).
    \item Een vergelijking met Netskope's probabilistische detectiemethoden.
    \item Een confusion matrix om de prestaties van de regex-detectieregels te evalueren.
\end{itemize}


% De risicoanalyse voor DLP (Data Loss Prevention) sluit aan bij de methodologie zoals beschreven in NIST SP 800-30 Rev. 1 en ISO 27005. 
% Volgens Fikri et al. (2019) bestaat een effectieve risicoanalyse uit een combinatie van impact- en waarschijnlijkheidsbeoordelingen, 
% waarbij elk risico wordt gekwantificeerd op basis van de ernst van de gevolgen en de waarschijnlijkheid van optreden. 
% Hierbij wordt rekening gehouden met factoren zoals vertrouwelijkheid, integriteit en beschikbaarheid van de gegevens.

% De tabel in deze studie categoriseert gegevensrisico’s op basis van een vijfpuntsschaal, 
% vergelijkbaar met de impact- en waarschijnlijkheidstabellen in NIST SP 800-30. 
% Net zoals in de methode van Fikri et al. (2019) wordt in deze analyse onderscheid gemaakt tussen:

% Impactniveau: De ernst van de schade die een datalek kan veroorzaken, variërend van "zeer laag" (minimale impact) tot "zeer hoog" (catastrofale gevolgen).
% Waarschijnlijkheidsniveau: De kans dat een bedreiging zich manifesteert, gebaseerd op historische data en systeemzwakheden.
% Een belangrijk verschil tussen deze aanpak en de methode uit NIST SP 800-30 is de specifieke focus op datatypes en hun combinaties. 
% Terwijl de methode van Fikri et al. (2019) zich richt op algemene risico’s binnen informatiesystemen, 
% hanteert deze studie een meer gedetailleerde indeling voor PII (Persoonlijk Identificeerbare Informatie), 
% authenticatiegegevens en bedrijfsgevoelige data, wat resulteert in een praktisch toepasbare DLP-risicobeoordeling.

% Door deze principes toe te passen, wordt een semi-kwantitatieve methode gehanteerd die niet alleen de potentiële impact van een datalek in kaart brengt, 
% maar ook de waarschijnlijkheid van misbruik door kwaadwillenden of interne fouten. 
% Dit maakt de risicoanalyse zowel consistent met NIST SP 800-30 als toepasbaar voor specifieke DLP-beleidscategorieën.

\section{\IfLanguageName{dutch}{Data en risicobeoordeling}{Data and Risk Assessment}}
\label{sec:data_risico}

\subsubsection{Waarschijnlijkheid van gevoelige gegevens}
Niet elk individueel gegevenselement vormt een even groot risico. Een e-mailadres op zich is minder gevoelig dan een naam gekoppeld aan een bankrekeningnummer. 
% De volgende tabel toont de gewichten toegekend aan verschillende datatypes en combinaties:

% In tabel \ref{tab:datatypes_risico} worden de waarschijnlijkheidsgewichten van verschillende datatypes en combinaties getoond. \textcite{al2019risk}
% De waarden variëren van 1 tot 5 en geven als volgt de risicograad aan:

Volgens het model van \textcite{AlFikri2019} vormen impact en waarschijnlijkheid de basis voor risico\-inschattingen. 
\gls{nist} \autocite{NIST800-30} en \gls{iso5} \autocite{IOSIEC2022} vormen de basis van dit model.
Het model kwantificeert risico's op een schaal van \textbf{0} tot \textbf{10}. 
Tabel~\ref{tab:datatypes_risico_uitgebreid} toont de impact- en waarschijnlijkheidsscores van verschillende datatypes en combinaties, 
met bijbehorende totale scores op een schaal van \textbf{1} tot en met \textbf{25}.
Deze schaal is meer geschikt voor de specifieke context van DLP-risico's binnen Evolane.
Tabel~\ref{tab:risico_thresholds} definieert de thresholds op basis van de gehanteerde schaal.

\begin{itemize}
    \item \textbf{1\--4: Zeer Laag Risico} Gegevens die op zichzelf weinig impact hebben bij een lek. Het verlies vormt geen directe bedreiging voor privacy of veiligheid, zoals een losse naam of publiek IP-adres.
    \item \textbf{5\--9: Laag Risico} Individuele gegevens met enige waarde voor aanvallers, zoals een openbaar e-mailadres zonder aanvullende context.
    \item \textbf{10\--14: Gemiddeld Risico} Combinaties van persoonsgegevens die identiteitsdiefstal kunnen vergemakkelijken, zoals naam + adres of naam + geboortedatum.
    \item \textbf{15\--19: Hoog Risico} Gevoelige informatie, zoals financiële gegevens, die een aanzienlijke impact kunnen hebben bij een lek, zoals bankrekeningnummers of vertrouwelijke bedrijfsinformatie.
    \item \textbf{20\--25: Zeer Hoog Risico} Kritieke gegevenscombinaties die grootschalige beveiligingsrisico's kunnen veroorzaken of ernstige schade kunnen aanrichten, zoals toegangscertificaten of gevoelige financiële informatie.
\end{itemize}
% \begin{table}[h]
%     \centering
%     \small
%     \begin{tabular}{|l|c|}
%     \hline
%     \textbf{Datatype/combinatie} & \textbf{Risicogewicht (1-5)} \\
%     \hline
%     \multicolumn{2}{|l|}{\textbf{Persoonlijke Identificeerbare Informatie (PII)}} \\
%     \hline
%     Naam alleen                 & 1 \\
%     Publiek IP-adres            & 1 \\
%     Naam + E-mailadres          & 2 \\
%     Naam + Telefoonnummer       & 2 \\
%     Naam + Geboortedatum        & 3 \\
%     Naam + BSN (NL) / NISS (BE) & 4 \\
%     Paspoortnummer              & 4 \\
%     Privaat IP-adres (intern)   & 2 \\
%     \hline
%     \multicolumn{2}{|l|}{\textbf{Betaalkaart \& PCI Gegevens}} \\
%     \hline
%     Bankrekeningnummer alleen   & 2 \\
%     Naam + Bankrekeningnummer   & 4 \\
%     Creditcardnummer            & 4 \\
%     Naam + Creditcardnummer     & 5 \\
%     CVV/CVC-code alleen         & 1 \\
%     CVV/CVC-code + Creditcardnummer & 5 \\
%     PIN-code + Kaartnummer      & 5 \\
%     \hline
%     \multicolumn{2}{|l|}{\textbf{Authenticatie \& Inloggegevens}} \\
%     \hline
%     E-mailadres alleen          & 1 \\
%     E-mailadres + Wachtwoord    & 5 \\
%     API-sleutel                 & 5 \\
%     JWT-token                   & 4 \\
%     SSH-private key             & 5 \\
%     \hline
%     \multicolumn{2}{|l|}{\textbf{Bedrijfsgevoelige Data}} \\
%     \hline
%     Klantnaam + Contractdetails & 4 \\
%     Klantnaam + Factuurgegevens & 5 \\
%     Interne projectinformatie   & 3 \\
%     Gevoelige e-mails (bijv. NDA's) & 4 \\
%     Configuratiebestanden met credentials & 5 \\
%     \hline
%     \end{tabular}
%     \caption{Waarschijnlijkheid van datatypes voor DLP-risico}
%     \label{tab:datatypes_risico}
% \end{table}


\begin{table}[h]
    \centering
    \small
    \scriptsize
    \begin{tabular}{|c|c|c|c|c|}
    \hline
    \textbf{Datatype/combinatie} & \textbf{Waarschijnlijkheid (1-5)} & \textbf{Impact (1-5)} & \textbf{Totaal Risico} & \textbf{Categorie} \\
    \hline
    \multicolumn{5}{|l|}{\textbf{Persoonsgegevens (\gls{pii})}} \\
    \hline
    Naam                        & 5 & 1 & 5  & Laag \\
    Publiek IP-adres            & 4 & 1 & 4  & Laag \\
    Naam + E-mailadres          & 5 & 3 & 15 & Hoog \\
    Naam + Telefoonnummer       & 4 & 3 & 12 & Gemiddeld \\
    Naam + Geboortedatum        & 4 & 4 & 16 & Hoog \\
    Naam + \gls{niss} (BE) & 3 & 5 & 15 & Hoog \\
    Paspoortnummer              & 2 & 5 & 10 & Gemiddeld \\
    Privaat IP-adres (intern)   & 3 & 2 & 6  & Laag \\
    \hline
    \multicolumn{5}{|l|}{\textbf{Betalingsgegevens (\gls{pci})}} \\
    \hline
    Bankrekeningnummer          & 3 & 2 & 6  & Laag \\
    Naam + Bankrekeningnummer   & 4 & 4 & 16 & Hoog \\
    Creditcardnummer            & 3 & 4 & 12 & Gemiddeld \\
    Naam + Creditcardnummer     & 3 & 5 & 15 & Hoog \\
    \gls{cvv}/\gls{cvc}-code    & 2 & 3 & 6  & Laag \\
    \gls{cvv} + Creditcardnummer      & 3 & 5 & 15 & Hoog \\
    PIN-code + Kaartnummer      & 2 & 5 & 10 & Gemiddeld \\
    \hline
    \multicolumn{5}{|l|}{\textbf{Authenticatie \& Inloggegevens}} \\
    \hline
    E-mailadres                 & 5 & 1 & 5  & Laag \\
    E-mailadres + Wachtwoord    & 5 & 5 & 25 & Zeer hoog \\
    \gls{apikey}                 & 4 & 5 & 20 & Zeer hoog \\
    \gls{jwt}                  & 3 & 4 & 12 & Gemiddeld \\
    \gls{ssh}-private key             & 3 & 5 & 15 & Hoog \\
    \hline
    \multicolumn{5}{|l|}{\textbf{Bedrijfsgevoelige Data}} \\
    \hline
    Klantnaam + Contractdetails & 4 & 4 & 16 & Hoog \\
    Klantnaam + Factuurgegevens & 4 & 5 & 20 & Zeer hoog \\
    Interne projectinformatie   & 4 & 3 & 12 & Gemiddeld \\
    Gevoelige e-mails (bv. \gls{nda}'s) & 3 & 4 & 12 & Gemiddeld \\
    Config-bestanden met credentials & 3 & 5 & 15 & Hoog \\
    Source code                 & 3 & 5 & 15 & Hoog \\
    \hline
    \end{tabular}
    \caption{Risico-analyse van datatypes voor DLP-configuratie binnen een Sales/IT-omgeving}
    \label{tab:datatypes_risico_uitgebreid}
\end{table}


\subsubsection{Toelichting op impact- en waarschijnlijkheidsscores}
\label{sec:toelichting_scores}

De impact en waarschijnlijkheid van datalekken zijn contextafhankelijk en kunnen variëren per organisatie. 
Om een zo objectief mogelijke inschatting te maken, zijn volgende bronnen gebruikt tijdens de risicoanalyse:


\paragraph{Bronnen en context}
\begin{itemize}
  \item \textbf{Rapporten/ Openbare bronnen:} \gls{cve}'s, \gls{owasp}, Verizon Data Breach Reports en IBM X-Force Threat Intelligence Reports.
  \item \textbf{Interne ervaring:} De workflow van een Security Engineer bij Evolane onderzoeken en bekijken welke datatypes vaak voorkomen in support-tickets, logs en \gls{saas}-omgevingen. 
  \item \textbf{PoC-metingen:} Tijdens dit onderzoek wordt gekeken naar hoe vaak welke datatypes voorkomen in datalekken (\gls{pii}, \gls{pci}, \gls{api}, enzovoort). 
  \item \textbf{Compliance:} Scores voor impact zijn vooral gebaseerd op de mogelijke gevolgen van datalekken volgens relevante wet- en regelgeving zoals de \gls{avg}, \gls{pcidss}, \gls{nis2} en \gls{iso}.
\end{itemize} 

\paragraph{Criteria voor waarschijnlijkheid (1 = zeer zeldzaam \-- 5 = zeer frequent)}

De waarschijnlijkheidsscores zijn gebaseerd op hoe vaak een bepaald datatype of combinatie voorkomt in de context van Evolane en andere organisaties. 
Hierbij wordt ook rekening gehouden met hoe makkelijk deze data te detecteren is met regex of andere methoden.

\begin{itemize}
  \item \textbf{1 (Zeer Laag):} Data komt praktisch nooit onbeschermd voor (bv.~\textit{biometrie, medische dossiers}).
  \item \textbf{2 (Laag):} Incidenteel in logs of support-tickets aangetroffen (bv.~\textit{\gls{cvv}-codes alleen in schermopnames}).
  \item \textbf{3 (Gemiddeld):} Regelmatig in testomgevingen of via shadow-IT — detecteerbaar met standaard regex (bv.~\textit{creditcardnummers}).
  \item \textbf{4 (Hoog):} Vaak in \textbf{e-mail} of \textbf{logs} aanwezig (bv.~\gls{naw}-gegevens).
  \item \textbf{5 (Zeer Hoog):} Data die \textbf{voortdurend} wordt verwerkt en gedeeld, vaak over meerdere kanalen (bv.~\textit{e-mailadressen, namen}).
\end{itemize}

\paragraph{Criteria voor impact (1 = verwaarloosbaar \-- 5 = catastrofaal)}

De impactscores zijn daarentegen gebaseerd op de mogelijke schade die een datalek kan veroorzaken, zowel financieel als operationeel.

\begin{itemize}
  \item \textbf{1 (Zeer Laag):} Geen wettelijke consequenties (bv.~\textit{Publiek toegankelijk IP-adres}).
  \item \textbf{2 (Laag):} Lage boetes of kleine reputatieschade (bv.~losse \gls{naw}-gegevens).
  \item \textbf{3 (Gemiddeld):} Reële kans op fraude of identiteitsdiefstal, gegevens kunnen gelinkt worden (bv.~creditcardnummer met mogelijks \gls{naw}, maar zonder \gls{cvv}).
  \item \textbf{4 (Hoog):} Hoge boetes en reputatieschade, operationele impact (bv.~\gls{naw} + bankrekeningnummer met \gls{cvv}, \gls{naw} + gevoelige klantdata).
  \item \textbf{5 (Zeer Hoog):} Maximale \gls{avg}/\gls{pcidss}-boetes, grootschalige bedrijfsstilstand (bv.~e-mail + wachtwoord, \gls{api}, in-memory klantdata).
\end{itemize}

Om de totale risicoscore te berekenen, worden de waarschijnlijkheids- en impactscores vermenigvuldigd.

\begin{equation}
    \text{Risicoscore} = \text{Waarschijnlijkheid} \times \text{Impact}
\end{equation}



Ter illustratie van bovenstaande criteria~\ref{tab:datatypes_risico_uitgebreid}, zie de volgende voorbeelden en de gebruikte gedachtengang:

\paragraph{Voorbeeldtoelichting per categorie}
\begin{itemize}
  \item \textbf{Naam + e-mail:}  
    \begin{itemize}
      \item \emph{Waarschijnlijkheid = 5}: deze combinatie is dagelijks terug te vinden in support-tickets en logs.
      \item \emph{Impact = 3}: combinatie van naam en e-mail kan leiden tot phishing-aanvallen, maar is niet direct schadelijk zonder verdere context.
    \end{itemize}
  \item \textbf{E-mail + wachtwoord:}
    \begin{itemize}
      \item \emph{Waarschijnlijkheid = 5}: één van de populairste doelwitten in datalek-archives.
      \item \emph{Impact = 5}: directe account-takeover, toegang tot gevoelige data en mogelijkheid tot reputatieschade.
    \end{itemize}
  \item \textbf{\gls{apikey}:}
    \begin{itemize}
      \item \emph{Waarschijnlijkheid = 4}: hardcoded in code, regelmatig te vinden in Git-repos en cloud-backups.
      \item \emph{Impact = 5}: volledige system-takeover, hoge operationele en financiële schade.
    \end{itemize}
\end{itemize}

% Risicoscore=Waarschijnlijkheid (1-5)×Impact (1-5)
% Je kan dan per datatype:
% Waarschijnlijkheid beoordelen (hoe vaak komt dit voor + hoe makkelijk te detecteren)
% Impact beoordelen (schade bij lek)

% % De scores voor waarschijnlijkheid en impact in Tabel~\ref{tab:datatypes_risico_uitgebreid} zijn gebaseerd op factoren zoals;
% % praktische ervaring binnen de context van Evolane en richtlijnen uit~\autocite{IOSIEC2022} en~\autocite{NIST800-30}.

% % \paragraph{Persoonlijke identificeerbare informatie (PII):} 
% % Losse persoonsgegevens zoals een naam of IP-adres zijn op zichzelf weinig schadelijk en komen vaak voor, 
% % waardoor hun waarschijnlijkheid relatief hoog ligt maar impact beperkt blijft (score 1–2). 
% % Zodra PII wordt gecombineerd — bijvoorbeeld naam + geboortedatum of naam + NISS — stijgt de impact significant omdat deze combinaties identiteitsfraude kunnen faciliteren. 
% % De waarschijnlijkheid wordt hierbij vaak iets lager ingeschat, aangezien dergelijke gegevens minder frequent samen voorkomen in vrij toegankelijke contexten.

% % \paragraph{Betaalkaart- en PCI-gegevens:}
% % Gegevens zoals bankrekening- of creditcardnummers hebben doorgaans een hogere impactscore vanwege de mogelijke financiële schade. 
% % Combinaties zoals naam + kaartnummer of CVV + creditcardnummer verhogen de kans op frauduleus gebruik. 
% % De waarschijnlijkheid is hier meestal iets lager (2–3), omdat deze data minder snel opduiken zonder beveiliging.

% % \paragraph{Authenticatie- en inloggegevens:}
% % Inlogcombinaties zoals e-mailadres + wachtwoord worden frequent aangetroffen in datalekken en vormen een direct risico op account takeovers. 
% % De waarschijnlijkheid is zeer hoog (5) en de impact potentieel catastrofaal (5).
% % API-sleutels en tokens worden in technische contexten vaak onderschat, maar vormen bij lekkage een ernstig risico op ongeautoriseerde toegang, 
% % wat hun impactscore rechtvaardigt.

% % \paragraph{Bedrijfsgevoelige data:}
% % Hoewel minder strikt gereguleerd dan persoonsgegevens, kan lekken van contractdetails, NDA’s of configuratiebestanden met credentials ernstige zakelijke gevolgen hebben. 
% % De waarschijnlijkheid is hier afhankelijk van hoe deze informatie wordt gedeeld (via e-mail, cloudopslag, ...), wat meestal resulteert in een score van 3–4. 
% % De impact is hoger bij strategische informatie of credentials (score 4–5).

% % \paragraph{Speciale categorieën:}
% % Bepaalde types gegevens zoals biometrische data of gezondheidsinformatie worden in deze context vermeld maar niet opgenomen in de tabel, 
% % omdat ze minder relevant zijn voor de huidige PoC-omgeving. Desondanks worden ze bij aanwezigheid standaard beschouwd als zeer gevoelig volgens de GDPR (impactscore 5).

% \textbf{Toelichting bij belangrijke combinaties:}
% \begin{enumerate}
%     \item \textbf{Naam + geboortedatum + IBAN:} Zeer gevoelig voor identiteitsfraude; valt onder \gls{avg}.
%     \item \textbf{E-mail + wachtwoord:} Hoog risico op account takeovers; vaak doelwit bij datalekken.
%     \item \textbf{API-sleutels en config-bestanden:} Vaak hardcoded in code, eenvoudig te lekken via cloudstorage.
%     \item \textbf{Creditcardgegevens:} Vallen onder \gls{pcidss}; verwerking en opslag zijn streng gereguleerd.
%     \item \textbf{Biometrie en gezondheid:} Speciale categorie onder \gls{avg}; impact bij lek is zeer hoog.
%     \item \textbf{Tokens (JWT/OAuth):} Directe toegang tot accounts/sessies bij lek; iets lagere waarschijnlijkheid.
% \end{enumerate}


Tabellen~\ref{tab:risico_thresholds} en~\ref{tab:risico_acties} geven de thresholds en acties weer die genomen worden op basis van de risicoscores~\ref{tab:datatypes_risico_uitgebreid}.
De thresholds~\ref{tab:risico_thresholds} zijn afgestemd samen met de opdrachtgever van Evolane.

\begin{table}[h]
    \centering
    \small
    \scriptsize
    \begin{tabular}{l c c c c}
        \toprule
        \textbf{Categorie risico} & \textbf{Low} & \textbf{Medium} & \textbf{High} & \textbf{Critical} \\
        \midrule
        % 1 - Zeer Laag     & $\leq$ 4   & 9   & 14  & $\geq$ 23 \\
        1\--4 \-- Zeer Laag     &  4   & 9   & 14  &  23 \\
        5\--9 \-- Laag          &  2   & 4   & 9   &  14 \\
        10\--14 \-- Gemiddeld     &  1   & 2   & 4   &  9 \\
        15\--19 \-- Hoog          &  1   & 2   & 4   &  9 \\
        20\--25 \-- Zeer Hoog     &  1   & 2   & 3   &  5 \\
        \bottomrule
    \end{tabular}
    \caption{Threshold per impactniveau voor risico-inschatting}
    \label{tab:risico_thresholds}
\end{table}

Verder zijn de acties~\ref{tab:risico_acties} die Netskope onderneemt, gebaseerd op het eindelijke doel van de DLP-regels. 
Tijdens de PoC~\ref{ch:proofofconcept} staan deze acties op \textbf{Alert} of \textbf{User Alert}.

\begin{table}[h]
    \centering
    \small
    \scriptsize
    \begin{tabular}{l l l}
        \toprule
        \textbf{Categorie risico} & \textbf{Severity} & \textbf{Action Taken} \\
        \midrule
        1\--14 (Zeer Laag-Gemiddeld) & Low      & Alert only \\
                                  & Medium   & Block \\
                                  & High     & Block \\
                                  & Critical & Block and escalate \\
        15\--25 (Hoog-Zeer Hoog)  & Low      & Block \\
                                  & Medium   & Block \\
                                  & High     & Block and escalate \\
                                  & Critical & Block, escalate \\
        \bottomrule
    \end{tabular}
    \caption{Actie per combinatie van impactniveau en risicoseverity}
    \label{tab:risico_acties}
\end{table}
% \textbf{Hier komt nog een MINDMAP Om te tonen hoe alle soorten datatypen met elkaar verbonden zijn.}

\section{\IfLanguageName{dutch}{Wetgeving en richtlijnen}{Legislation and Guidelines}}
\label{sec:risicoanalyse_wetgevingen}

Aansluitend op hoofdstuk~\ref{ch:stand-van-zaken}, worden in deze sectie de relevante wetgeving en richtlijnen besproken die van toepassing zijn op de risicoanalyse. 
Deze wetgeving en richtlijnen tonen aan waar in het DLP-model de risicoanalyse en de implementatie van de DLP-regels aan bod komen.

\subsection{\IfLanguageName{dutch}{Algemene Verordening Gegevensbescherming,}{GDPR}}
\label{sec:avg}

Volgens de \gls{avg} moet de verwerking van persoonsgegevens voldoen aan pseudonimisering en encryptie om de vertrouwelijkheid, 
integriteit, beschikbaarheid en veerkracht van verwerkingssystemen te waarborgen \autocite{ICSA2018}.
\textcite{Netskope2025Encryption, Netskope2022Encryption} biedt deze principes aan in hun Cloud Security Platform.
De encryptie gebeurt via \textbf{AES-256} encryptie en een \textbf{FIPS 140\--2 Level 3}-gecertificeerde KMS, met de optie om een on-premises HSM te gebruiken.
Verder heeft \textcite{Netskope2023GDPR} een hoop vooraf gedefinieerde DLP-regels~\ref{subsubsec:vooraf-gedefinieerde-regels-literatuurstudie} die samen voldoen aan de \gls{avg}.

% De Algemene Verordening Gegevensbescherming (AVG/GDPR) reguleert de bescherming van persoonsgegevens in de EU. 
% Zij bepaalt dat persoonsgegevens verwerkt worden met passende technische en organisatorische beveiliging (artikel 32 GDPR), 
% zoals pseudonimisering en encryptie om de vertrouwelijkheid, integriteit, beschikbaarheid en veerkracht van verwerkingssystemen te waarborgen


\subsection{\IfLanguageName{dutch}{NIS2-richtlijn}{NIS2 Directive}}
\label{sec:nis2_richtlijn}
% \autocite{nis2directive} 
Zoals besproken in de literatuurstudie~\ref{sec:nationale-europese-richtlijnen-literatuurstudie}, 
gaat de \gls{nis2}-richtlijn over de beveiliging van netwerken en informatiesystemen binnen de Europese Unie.
\textcite{Netskope2024NIS2} ondersteunt organisaties bij het naleven van deze richtlijn, 
door maatregelen aan te bieden rond \textbf{risicoanalyse}, \textbf{incidentenbeheer}, \textbf{toegangscontrole} en \textbf{continue monitoring}.
Evolane gebruikt cloudgebaseerde oplossingen voor het hosten van netwerken en informatiesystemen, wat betekent dat de netwerken en informatiesystemen van de organisatie zich extern bevinden.
Hierdoor is \gls{nis2} relevant voor de \textbf{endpoints} die confidentiële data verwerken, waarop de \textit{Netskope Client} geïnstalleerd is.
% Evolane moet Netskope dan ook vertrouwen voor de implementatie en het beheer van de juiste DLP-regels.
Een risicobeoordeling rond \gls{nis2} omvat de volgende elementen:

\begin{itemize}
    \item \textbf{Identificatie van netwerken en informatiesystemen} Dit is reeds voltooid door Evolane.
    \item \textbf{Risicoanalyse} Netskope laat toe risico's op confidentiële gegevens te identificeren via DLP-regels.
    \item \textbf{Beveiligingsmaatregelen} Encryptie, toegangscontrole en continue monitoring zijn reeds in gebruik binnen \textcite{Netskope2025Encryption, Netskope2022Encryption}.
    \item \textbf{Incidentenbehandeling} Netskope biedt ondersteuning bij detectie, mitigatie en rapportage van incidenten.
\end{itemize}

De focus van dit onderzoek ligt op de risicoanalyse en het opzetten van concrete DLP-regels binnen Netskope. Thresholds voor acties zijn opgenomen in tabellen~\ref{tab:risico_thresholds} en~\ref{tab:risico_acties}.

\subsection{\IfLanguageName{dutch}{ISO/IEC 27001}{ISO 27001}}
\label{sec:iso_27001}
% Vermelden  waar DLP vereist is in ISO27001, namelijk controls Annex A - A.5.36 Data Loss prevention , maar ook relevante controls A.5.12 Classification of InformationA.5.13 Labeling of Information
% \textcite{ISO2022}

Om organisaties te helpen bij het opzetten, implementeren en verbeteren van een Information Security Management System (\gls{isms}),
is \gls{iso}:\autocite{ISO2022} een internationale standaard hierbij.
In aansluiting op sectie~\ref{sec:iso-literatuurstudie}, zal hier de focus liggen op de relevantie van DLP binnen \gls{iso}.
% DLP is vereist binnen de \gls{iso}:\autocite{ISO2022} standaard, specifiek in de volgende controles:
In de 2022-herziening van ISO/IEC 27001 zijn onder Annex A verschillende technische controls toegevoegd die direct van toepassing zijn op Data Loss Prevention (DLP).  
\textcite{Netskope2024ISO} biedt voor al deze controls concrete implementatiemogelijkheden:

\begin{itemize}
  \item \textbf{A.5.12 Classification of Information}  
    \textcite{Netskope2023AI, NetskopeCatDef2025, Netskope2025CreateProfiles} kan gelabelde-, geclassificeerde informatie automatisch herkennen en hieraan policies koppelen. 
    Bovendien biedt de DLP-oplossing fingerprinting en \gls{nlp} aan om ongeclassificeerde data te classificeren en zo te beschermen. 

  \item \textbf{A.5.13 Labelling of Information}  
    Netskope biedt de mogelijkheid om labels toe te kennen aan data, zie DLP-regels in tabellen~\ref{tab:custom-dlp-profielen},~\ref{tab:custom-be-id},~\ref{tab:custom-be-iban},~\ref{tab:custom-be-iban-terms},~\ref{tab:custom-eu-card} en~\ref{tab:custom-eu-card-terms}

  \item \textbf{A.8.11 Data Masking}  
    Voor \textit{data-at-rest} biedt \textcite{Netskope2022Encryption, Netskope2025Encryption} masking-functionaliteit waarmee gevoelige inhoud geobfusceerd kan worden in de console,
    zonder dat de originele data verloren gaat.

  \item \textbf{A.8.12 Data Loss Prevention}  
    Direct gerelateerd aan dit onderzoek, deze controle vereist de implementatie van een DLP-oplossing om gegevenslekken te voorkomen.
    \textcite{Netskope2025DLP} biedt deze DLP-oplossing aan.
\end{itemize}

% Met deze mapping sluit Netskope aan aan de vereisten uit \gls{iso}:2022 Annex A.
\subsection{\IfLanguageName{dutch}{Payment Card Industry Data Security Standard}{PCI DSS}}
\label{sec:pci_dss}

% \textcite{Netskope2024PCIDSS} \gls{pcidss}

Aansluitend op de sectie~\ref{sec:pcidss-literatuurstudie}, is de \gls{pcidss} een set van beveiligingsstandaarden die organisaties verplicht om de veiligheid van kaartgegevens (\gls{pan}, \gls{cvv}/\gls{sad}) te beschermen.
\textcite{Netskope2024PCIDSS} biedt een uitgebreide DLP-oplossing die organisaties helpt te voldoen aan de vereisten van \gls{pcidss}, waar Netskope zelf ook aan voldoet.
Relevante vereisten staan vermeld in volgende lijst:

{\small
  \begin{itemize}
    \item \textbf{1. Install and maintain Network Security Controls}  
          Netskope One Private Access (\gls{npa}) biedt een geavanceerde \gls{ztna}-oplossing die \gls{vpn}'s vervangt door toegangscontrole op basis van gebruikersidentiteit, apparaathouding en risicoprofielen \autocite{NetskopePrivateAccess2025}. 
    \item \textbf{3. Protect Stored Account Data}  
          \textcite{Netskope2022Encryption, Netskope2025Encryption} encryptieert en tokenizeert confidentiële data, zoals \gls{pan} en \gls{cvv}/\gls{sad}, om deze te beschermen tegen ongeautoriseerde toegang.
    \item \textbf{4. Protect Cardholder Data with Strong Cryptography During Transmission Over Open, Public Networks}  
          Zoals vermeld in sectie~\ref{sec:avg}, wordt \textbf{AES-256} encryptie gebruikt \autocite{Netskope2022Encryption, Netskope2025Encryption}.
    \item \textbf{5. Protect All Systems and Networks from Malicious Software}  
          SkopeAI biedt bescherming tegen malware door gebruik te maken van \gls{ml} om verdachte activiteiten te detecteren en te blokkeren \autocite{Netskope2024PCIDSS}.
    \item \textbf{7. Restrict Access to Cardholder Data by Business Need to Know}  
          \textcite{Netskope2024PCIDSS} \gls{swg} gebruikt \gls{sso} en \gls{mfa} om gebruikers te authenticeren en autoriseren op het webplatform. 
    \item \textbf{8. Identify Users and Authenticate Access to System Components}  
          \textcite{Netskope2024PCIDSS} ondersteunt rol-gebaseerde toegangscontrole en continue monitoring voor incidentenbeheer.
    \item \textbf{9. Restrict Physical Access to Cardholder Data}  
          \textcite{Netskope2025Endpoint} biedt Endpoint Security aan. 
    \item \textbf{10. Log and Monitor All Access to System Components and Cardholder Data}  
          \textcite{Netskope2022SSPM} (zie~\ref{fig:netskope_incidenten})
    \item \textbf{12. Support Information Security with Organizational Policies and Programs} 
          \textcite{Netskope2022SSPM} \gls{sspm} monitort \gls{iaas} en \gls{saas} platformen.
  \end{itemize}
}

\subsection{\IfLanguageName{dutch}{Digital Operational Resilience Act}{DORA}} % Nog toe te voegen aan literatuurstudie
\label{sec:dora}

De Digital Operational Resilience Act (\gls{dora}) is een Europese verordening die de financiële sector moet versterken tegen digitale storingen en cyberdreigingen \autocite{EIOPA2025}.
\gls{dora} stelt regels voor rond ICT-risicobeheer, waaronder het inrichten van detectiemechanismen voor verdachte ICT-activiteiten en het formaliseren van herstelplannen bij incidenten.
\textcite{NetskopeCompliance2025} is compliant met \gls{dora} door het bieden van oplossingen voor \textbf{Zero Trust}, \textbf{Network Segmentation}, \textbf{Data Protection} en \textbf{Incident Response}.
\gls{dora} is vergelijkbaar met voorgaande secties~\ref{sec:nis2_richtlijn},~\ref{sec:pci_dss}.

% STUDIE VOOR IEDEREEN https://chatgpt.com/c/6829e2e1-262c-8010-8b7a-a362d5f19a38 !!!!!!!!!!!!!!!!!!!!!!!!!!!!!!!!!!!!!!!!!!!!!!!!!!!!!!!!!!!!!!!!!!!!!!!!!!!!!!!!!!!!!!!!!

\subsection{\IfLanguageName{dutch}{Schrems II}{Schrems II}}
\label{sec:schrems_ii}

De \textcite{EuropeanCommission2023} heeft het EU-VS Privacy Shield vervangen door het EU-US Data Privacy Framework, naar aanleiding van de Schrems II-uitspraak van het Europese Hof van Justitie (C-311/18, 2020).
% Dit vanwege het EU-VS Privacy Shield onvoldoende bescherming bood tegen surveillance door de VS, waardoor de overdracht van persoonsgegevens naar de VS in strijd was met de \gls{avg}.
Sindsdien mogen persoonsgegevens enkel naar landen buiten de EU worden overgedragen indien zij onder de \gls{avg} een gelijkwaardig beschermingsniveau bieden.
Aangezien Netskope reeds voldoet aan de \gls{avg}~\ref{sec:avg}, is het ook compliant met de Schrems II-uitspraak.


% \section{Probabilistische detectie in Netskope}
% Netskope gebruikt een combinatie van regex-detectie en machine learning om waarschijnlijkheidsscores toe te kennen aan mogelijke datalekken \autocite{Netskope2023}. Dit werkt als volgt:

% \begin{itemize}
%     \item Pattern matching met regex geeft een score op basis van het aantal matches.
%     \item Contextuele analyse houdt rekening met omliggende woorden om false positives te reduceren.
%     \item Confidence levels van Netskope geven een confidence score toe op basis van detectiepatronen (bijv. een bankrekeningnummer dat altijd 16 cijfers heeft versus een willekeurige numerieke string).
% \end{itemize}

% De waarschijnlijkheid per detectie wordt gemodelleerd als volgt:

% \begin{equation}
%     P(\text{Datalek}) = P(\text{Match}) \times P(\text{Context}) \times P(\text{Verificatie})
% \end{equation}

% waar:
% \begin{itemize}
%     \item \( P(\text{Match}) \) Kans dat de regex correct een patroon herkent.
%     \item \( P(\text{Context}) \) Kans dat omliggende context de match versterkt (bijv. het woord "rekeningnummer" naast een reeks cijfers).
%     \item \( P(\text{Verificatie}) \) Kans dat de detectie correct wordt geclassificeerd als gevoelige data.
% \end{itemize}

% % % % % % \section{\IfLanguageName{dutch}{Evaluatie van detectieprestaties}{Evaluation of Detection Performance}}
% % % % % % \label{sec:evaluatie_detectie}

% % % % % % \subsection{Confusion Matrix voor regex-detectie}
% % % % % % Om de effectiviteit van de regex-gebaseerde detectie te beoordelen, gebruiken we een confusion matrix. Dit laat zien hoe vaak een detectie correct of foutief is:

% % % % % % \textbf{Hier ben ik nog niet tevreden mee. Ik zou graag nog alle tabellen uitleggen en de formules tijdens de poc implementeren.}

% % % % % % \begin{table}[h]
% % % % % %     \centering
% % % % % %     \small
% % % % % %     \scriptsize
% % % % % %     \begin{tabular}{|c|c|c|}
% % % % % %         \hline
% % % % % %         \textbf{} & \textbf{Werkelijke Gevoelige Data} & \textbf{Geen Gevoelige Data} \\ \hline
% % % % % %         \textbf{Gedetecteerd als gevoelig} & True Positives (TP) & False Positives (FP) \\ \hline
% % % % % %         \textbf{Niet gedetecteerd als gevoelig} & False Negatives (FN) & True Negatives (TN) \\ \hline
% % % % % %     \end{tabular}
% % % % % %     \caption{Confusion Matrix voor regex-detectie}
% % % % % %     \label{tab:confusion_matrix}
% % % % % % \end{table}

% % % % % % De belangrijkste prestatiemetrics zijn:
% % % % % % \begin{itemize}
% % % % % %     \item Precision \( = \frac{TP}{TP + FP} \) Hoe vaak een detectie correct was.
% % % % % %     \item Recall \( = \frac{TP}{TP + FN} \) Hoeveel gevoelige data correct wordt gedetecteerd.
% % % % % %     \item F1-score \( = 2 \times \frac{\text{Precision} \times \text{Recall}}{\text{Precision} + \text{Recall}} \) Gebalanceerde prestatiemeting.
% % % % % % \end{itemize}



% --------------------------------------------------------------------------------------------------------------------------------------------------------------------------------------
% Specificiteit, positief voorspellende waarde, negatief voorspellende waarde, likelihood ratio's, enz. kunnen ook worden berekend om de prestaties van de detectie te evalueren \autocite{Fawcett2006}.
% Sensitivity, specificity, predictive values, and likelihood ratios \autocite{Fawcett2006}.
% PPV betekent positief voorspellende waarde, hiermee wordt bedoeld hoeveel van de positieve voorspellingen daadwerkelijk positief zijn. 
% NPV betekent negatief voorspellende waarde, hiermee wordt bedoeld hoeveel van de negatieve voorspellingen daadwerkelijk negatief zijn..
% https://www.researchgate.net/publication/49650721_Sensitivity_specificity_predictive_values_and_likelihood_ratios/figures?lo=1 
% --------------------------------------------------------------------------------------------------------------------------------------------------------------------------------------



% \section{\IfLanguageName{dutch}{Resultaten en Conclusie}{Results and Conclusion}}
% \label{sec:resultaten_conclusie}

% \subsection{Resultaten van regex-evaluatie}
% Na het uitvoeren van de regex-tests op verschillende datasets, zijn de volgende resultaten behaald:

% \begin{itemize}
%     \item \textbf{Precision:} 85\% De regex-detectie leverde relatief weinig vals positieven op.
%     \item \textbf{Recall:} 72\% Een deel van de gevoelige data wordt niet correct gedetecteerd.
%     \item \textbf{F1-score:} 78\% De balans tussen precision en recall toont ruimte voor verbetering.
% \end{itemize}

% \subsection{Verbeterpunten voor detectie}
% Op basis van de confusion matrix en prestatiemetrics zijn verbeteringen mogelijk:
% \begin{itemize}
%     \item Optimalisatie van regex-patterns om vals negatieven te verminderen.
%     \item Betere contextdetectie binnen Netskope om vals positieven te verlagen.
%     \item Gebruik van machine learning-ondersteuning voor complexere datacombinaties.
% \end{itemize}

% \subsection{Conclusie}
% De evaluatie laat zien dat regex-gebaseerde detectie een effectieve methode is voor DLP, maar dat probabilistische modellen zoals die van Netskope een belangrijk voordeel bieden in contextuele en adaptieve detectie. Toekomstig werk kan zich richten op het verbeteren van recall door geavanceerde patroonherkenning en contextgebaseerde AI-modellen te gebruiken.







% \chapter{\IfLanguageName{dutch}{Risicoanalyse}{Risk Analysis}}%
% \label{ch:risicoanalyse}

% % https://www.scribbr.nl/modellen/risicoanalyse/

% % https://www.cyberpilot.io/cyberpilot-blog/what-a-risk-analysis-is-and-a-step-by-step-guide-to-guide-to-do-one-free-template?utm_source=chatgpt.com

% % https://docs.trendmicro.com/en-us/documentation/article/trend-micro-e-mail-security-online-help-dlp-compliance-templ?utm_source=chatgpt.com

