%%=============================================================================
%% Samenvatting
%%=============================================================================

% TODO: De "abstract" of samenvatting is een kernachtige (~ 1 blz. voor een
% thesis) synthese van het document.
%
% Een goede abstract biedt een kernachtig antwoord op volgende vragen:
%
% 1. Waarover gaat de bachelorproef?
% 2. Waarom heb je er over geschreven?
% 3. Hoe heb je het onderzoek uitgevoerd?
% 4. Wat waren de resultaten? Wat blijkt uit je onderzoek?
% 5. Wat betekenen je resultaten? Wat is de relevantie voor het werkveld?
%
% Daarom bestaat een abstract uit volgende componenten:
%
% - inleiding + kaderen thema
% - probleemstelling
% - (centrale) onderzoeksvraag
% - onderzoeksdoelstelling
% - methodologie
% - resultaten (beperk tot de belangrijkste, relevant voor de onderzoeksvraag)
% - conclusies, aanbevelingen, beperkingen
%
% LET OP! Een samenvatting is GEEN voorwoord!

%%---------- Nederlandse samenvatting -----------------------------------------
%
% TODO: Als je je bachelorproef in het Engels schrijft, moet je eerst een
% Nederlandse samenvatting invoegen. Haal daarvoor onderstaande code uit
% commentaar.
% Wie zijn bachelorproef in het Nederlands schrijft, kan dit negeren, de inhoud
% wordt niet in het document ingevoegd.

\IfLanguageName{english}{%
\selectlanguage{dutch}
\chapter*{Samenvatting}
% \lipsum[1-4]
\selectlanguage{english}
}{}

%%---------- Samenvatting -----------------------------------------------------
% De samenvatting in de hoofdtaal van het document

\chapter*{\IfLanguageName{dutch}{Samenvatting}{Abstract}}

De bescherming van vertrouwelijke bedrijfsgegevens vormt een cruciale uitdaging in de digitale wereld, 
vooral binnen de Belgische juridische context. 
Deze bachelorproef richt zich op de ontwikkeling en implementatie van een op maat gemaakte Netskope-gebaseerde Data Leakage Prevention (DLP) oplossing, 
specifiek afgestemd op de Belgische regelgeving. 
De hoofdonderzoeksvraag is hoe zo'n oplossing effectief kan worden ontworpen en toegepast om zowel te voldoen aan technische en juridische eisen. 
Met behulp van een combinatie van een literatuurstudie en een praktische Proof of Concept (PoC) worden de mogelijkheden van Netskope beoordeeld in het identificeren, 
beheersen en voorkomen van datalekken in een realistische testomgeving. Hierbij wordt rekening gehouden met wettelijke kaders zoals de 
Algemene Verordening Gegevensbescherming (AVG), de NIS2-richtlijn, evenals technische normen zoals PCI DSS en ISO 27001. 
De PoC zal worden uitgevoerd met realistische datascenario's om de oplossing te evalueren. Het onderzoek levert een werkend prototype van een DLP-oplossing op, 
samen met praktische aanbevelingen voor bedrijven die hun gevoelige data beter willen beschermen. 
De conclusie laat zien dat een goed uitgewerkte DLP-oplossing niet alleen helpt
om aan de wetgeving te voldoen, maar ook zorgt voor een betrouwbare bescherming tegen datalekken, zelfs in ingewikkelde bedrijfsomgevingen.
% Hier schrijf je de samenvatting van je voorstel, als een doorlopende tekst van één paragraaf. Let op: dit is geen inleiding, maar een samenvattende tekst van heel je voorstel met inleiding (voorstelling, kaderen thema), probleemstelling en centrale onderzoeksvraag, onderzoeksdoelstelling (wat zie je als het concrete resultaat van je bachelorproef?), voorgestelde methodologie, verwachte resultaten en meerwaarde van dit onderzoek (wat heeft de doelgroep aan het resultaat?).
% \lipsum[1-4]
