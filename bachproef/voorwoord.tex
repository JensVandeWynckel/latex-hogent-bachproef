%%=============================================================================
%% Voorwoord
%%=============================================================================

\chapter*{\IfLanguageName{dutch}{Woord vooraf}{Preface}}%
\label{ch:voorwoord}

%% TODO:
%% Het voorwoord is het enige deel van de bachelorproef waar je vanuit je
%% eigen standpunt (``ik-vorm'') mag schrijven. Je kan hier bv. motiveren
%% waarom jij het onderwerp wil bespreken.
%% Vergeet ook niet te bedanken wie je geholpen/gesteund/... heeft

% Het bevat vaak een korte beschrijving van het onderwerp, de aanleiding, persoonlijke ervaringen, en eventueel bedankjes aan betrokkenen

% \lipsum[1-2]

Deze bachelorproef markeert het eindpunt van mijn opleiding en was tegelijk een unieke kans om mijn academische kennis toe te passen in een reële bedrijfscontext.
Dankzij de samenwerking met Evolane kreeg ik de kans om een onderzoek uit te voeren naar een Data Loss Prevention-oplossing op basis van Netskope, afgestemd op Belgische en Europese regelgeving.
% Dit project gaf mij de kans om mijn technische kennis over cloudbeveiliging en DLP te verdiepen, terwijl ik inzicht kreeg in hoe IT-beveiliging en wetgeving elkaar beïnvloeden.
Dit project gaf mij de kans om mij te verdiepen in cloudbeveiliging en Data Loss Prevention (DLP), terwijl ik inzicht verwierf in hoe IT-beveiliging en wetgeving elkaar beïnvloeden.
De proof of concept werd uitgevoerd binnen een realistische omgeving van Evolane, wat zorgde voor een waardevolle en uitdagende ervaring.

Mijn oprechte dank gaat uit naar mijn promotor, Dhr. T. Desmedt, voor zijn begeleiding en ondersteuning gedurende dit project. 
Verder wil ik mijn copromotor, Dhr. M. Cremers, bedanken voor zijn feedback en inzichten die hebben bijgedragen aan het verbeteren van dit werk.
Tenslotte wil ik Dhr. K. Haeck bedanken om mij de kans te geven om dit project uit te voeren bij Evolane en voor zijn waardevolle input tijdens het onderzoek.

\vspace{1em}
\noindent
\textit{Jens Van de Wynckel} \\
\textit{Beveren-Kruibeke-Zwijndrecht, 30 mei 2025}

% % Evolane heeft mij de kans gegeven om dit project uit te voeren, aangezien zij op zoek waren naar een DLP-oplossing die aan hun behoeften voldeed.
% Dit project gaf mij de kans om mijn technische kennis over cloudbeveiliging en DLP te verdiepen, terwijl ik inzicht kreeg in hoe IT-beveiliging en wetgeving elkaar beïnvloeden.
% % De focus lag op het ontwerpen, implementeren en evalueren van een DLP-oplossing die voldoet aan wetgeving, zoals de \gls{avg}, \gls{nis2} en \gls{pcidss}.
% De samenwerking met Evolane, waar ik de proof of concept mocht uitvoeren in een realistische bedrijfsomgeving, maakte deze ervaring extra waardevol.
% Tijdens de evaluatie van de DLP-oplossing heb ik me vooral gericht op de correctheid van de Netskope-agent en de effectiviteit van de detectie van gevoelige data.

% Hierbij heb ik gebruik gemaakt van vooraf gedefinieerde DLP-regels en eigen gedefinieerde regels, om deze dan te evalueren op basis van een e-maildataset.


% Deze bachelorproef markeert het eindpunt van mijn opleiding en was tegelijk een unieke kans om mijn academische kennis toe te passen in een reële bedrijfscontext. 
% Dankzij de samenwerking met Evolane kreeg ik de kans om een onderzoek uit te voeren naar een Data Loss Prevention-oplossing op basis van Netskope, afgestemd op Belgische en Europese regelgeving.

% Tijdens dit traject kon ik mijn technische vaardigheden in cloudbeveiliging en data protectie verdiepen, maar minstens even belangrijk: 
% ik leerde hoe IT-beveiliging en juridische kaders elkaar beïnvloeden in de praktijk. De proof of concept werd uitgevoerd binnen de infrastructuur van Evolane, wat zorgde voor een uitdagende maar bijzonder leerrijke ervaring.

% Ik ben dankbaar voor de begeleiding die ik mocht ontvangen van mijn promotor en de experts binnen Evolane. 
% Hun feedback en inzichten hebben een grote rol gespeeld in het slagen van dit project. Deze bachelorproef heeft mijn interesse in cybersecurity alleen maar versterkt, en vormt voor mij een solide basis om verder te groeien binnen dit domein.
% \supervisor{Dhr. T. Desmedt}
% \cosupervisor{Dhr. M. Cremers}